% \iffalse meta-comment
%
% Copyright (C) 2023 by Zhang Tingxuan <alphaztx@163.com>
%
% This work may be distributed and/or modified under the
% conditions of the LaTeX Project Public License, either
% version 1.3c of this license or (at your option) any later
% version. The latest version of this license is in:
%
%   http://www.latex-project.org/lppl.txt
%
% and version 1.3 or later is part of all distributions of
% LaTeX version 2005/12/01 or later.
%
% This work has the LPPL maintenance status `maintained'.
%
% The Current Maintainer of this work is Zhang Tingxuan.
%
% This work consists of the files physics2-code.dtx,
%           and the derived files physics2.ins,
%                                 physics2.sty,
%                                 phy-<module>.sty,
%                                 physics2-code.pdf,
%                             and README.md.
%
%<*internal>
\iffalse
%</internal>
%
%<*readme>
# The `physics2` package

This package defines commands for typesetting math formulae faster and
more simply. `physics2` is a modularized package, each module provides its
own function while `physics2.sty` itself only provides very few functions.
Therefore you can load modules separately after loading `physics2`.

Modules of `physics2` provide the following supports:

- Automatic braces;
- Dirac bra-ket notation;
- Easy way to typeset diagonal matrices and matrices with similar entries;
- Double cross and double dot (binary) operators for tensors;

**Welcome to pull requests
[here](https://github.com/AlphaZTX/physics2/pulls).**

## License

This work may be distributed and/or modified under the conditions of the
[LaTeX Project Public License](http://www.latex-project.org/lppl.txt),
either version 1.3c of this license or (at your option) any later version.

This work has the LPPL maintenance status `maintained`.

The Current Maintainer of this work is
[Zhang Tingxuan](https://www.ctan.org/author/zhang-tx).

%</readme>
%
%<*internal>
\fi
\begingroup
  \def\NameOfLaTeXe{LaTeX2e}
\expandafter\endgroup\ifx\NameOfLaTeXe\fmtname\else
\csname fi\endcsname
%</internal>
%
%<*install>
\input docstrip.tex
\keepsilent

\preamble

  Copyright (C) 2023 by Zhang Tingxuan <alphaztx@163.com>

  This work may be distributed and/or modified under the
  conditions of the LaTeX Project Public License, either
  version 1.3c of this license or (at your option) any later
  version. The latest version of this license is in:

    http://www.latex-project.org/lppl.txt

  and version 1.3 or later is part of all distributions of
  LaTeX version 2005/12/01 or later.
 
  This work has the LPPL maintenance status `maintained'.
 
  The Current Maintainer of this work is Zhang Tingxuan.

  This work consists of the files physics2-code.dtx,
            and the derived files physics2.ins,
                                  physics2.sty,
                                  phy-<module>.sty,
                                  physics2-code.pdf,
                              and README.md.

\endpreamble

\generate{
  \usedir{tex/latex/physics2}
    \file{physics2.sty}          {\from{physics2-code.dtx}{package}}
    \file{phy-ab.sty}            {\from{physics2-code.dtx}{ab}}
    \file{phy-ab.braket.sty}     {\from{physics2-code.dtx}{ab.braket}}
    \file{phy-braket.sty}        {\from{physics2-code.dtx}{braket}}
    \file{phy-diagmat.sty}       {\from{physics2-code.dtx}{diagmat}}
    \file{phy-doubleprod.sty}    {\from{physics2-code.dtx}{doubleprod}}
    \file{phy-xmat.sty}          {\from{physics2-code.dtx}{xmat}}
    \file{phy-ab.legacy.sty}     {\from{physics2-code.dtx}{ab.legacy}}
    \file{phy-bm-um.legacy.sty}  {\from{physics2-code.dtx}{bm-um.legacy}}
    \file{phy-nabla.legacy.sty}  {\from{physics2-code.dtx}{nabla.legacy}}
    \file{phy-op.legacy.sty}     {\from{physics2-code.dtx}{op.legacy}}
    \file{phy-qtext.legacy.sty}  {\from{physics2-code.dtx}{qtext.legacy}}
%</install>
%<*internal>
  \usedir{source/latex/physics2}
    \file{physics2.ins}          {\from{physics2-code.dtx}{install}}
%</internal>
%<*install>
  \usedir{doc/latex/physics2}
  \nopreamble\nopostamble
    \file{README.md}             {\from{physics2-code.dtx}{readme}}
}

\obeyspaces
\Msg{**********************************************************}
\Msg{*                                                        *}
\Msg{*  To finish the installation you have to move the       *}
\Msg{*  following files into a directory searched by TeX:     *}
\Msg{*                                                        *}
\Msg{*                    physics2.sty,                       *}
\Msg{*  (for all modules) phy-<module>.sty.                   *}
\Msg{*                                                        *}
\Msg{*  The recommended directory is                          *}
\Msg{*    TEXMFDIST/tex/latex/physics2                        *}
\Msg{*                                                        *}
\Msg{*  To produce the documentation, run the file            *}
\Msg{*  physics2-code.dtx through XeLaTeX or LuaLaTeX.        *}
\Msg{*  XeLaTeX is recommended.                               *}
\Msg{*                                                        *}
\Msg{*  Happy TeXing!                                         *}
\Msg{*                                                        *}
\Msg{**********************************************************}

\endbatchfile
%</install>
%
%<*internal>
\fi
%</internal>
%
%<*driver>
\ProvidesFile{fixdif.dtx}[2023/03/20 (c) Copyright 2022-2023 by Zhang Tingxuan]
\documentclass{l3doc}
\usepackage{doc}
\hypersetup{hidelinks}
\usepackage{color}
\definecolor{pkgcolor}{RGB}{0,120,80}
\definecolor{moducolor}{RGB}{0,0,150}
\definecolor{gray}{rgb}{.5,.5,.5}
\DeclareRobustCommand\pkg[1]{\texorpdfstring{\textcolor{pkgcolor}{\textsf{#1}}}{“#1”}}
\DeclareRobustCommand\modu[1]{\texorpdfstring{\textcolor{moducolor}{\textsf{#1}}}{“#1”}}
\DeclareRobustCommand\note{\textbf{Note:} }
\usepackage{amsmath}
\usepackage{fontspec}
\setmonofont{Latin Modern Mono Light}
\usepackage{fancyvrb}
\fvset{gobble=2,xleftmargin=2em}
\usepackage{physics2}
\usephysicsmodule{ab}
\makeatletter
\@addtoreset{CodelineNo}{section}
\newbox\@tempboxb
\DeclareDocumentCommand\macrodef{O{4em}O{.6}mo+m}{\leavevmode
\ensuremath{%
  \setbox\@tempboxa\hbox to #1 {\hss\cs{#3}}%
  \setbox\@tempboxb\vbox{\hsize=#2\textwidth
    \textwidth=\hsize
    \leftskip=\z@
    \parindent=\z@
    \textbf{begindef}\IfValueT{#4}{ \textcolor{gray}{(#4)}}\par
    \begingroup\leftskip2em\parindent-1em#5\par\endgroup
    \textbf{enddef}%
  }%
  \dp\@tempboxb=\dimexpr\ht\@tempboxb-\ht\@tempboxa\relax
  \ht\@tempboxb=\ht\@tempboxa
  \box\@tempboxa\leftarrow\box\@tempboxb
}}
\protected\long\def\macroifelse#1#2#3{\par
  \textbf{if} #1 \textbf{then}\par
  \begingroup\advance\leftskip1em#2\par\endgroup
  \textbf{else}\par
  \begingroup\advance\leftskip1em#3\par\endgroup
  \textbf{endif}\par
}
\protected\long\def\macroif#1#2{\par
  \textbf{if} #1 \textbf{then}\par
  \begingroup\advance\leftskip1em#2\par\endgroup
  \textbf{endif}\par
}
\protected\def\macrocr{\par}
\newdimen\ttwd
\protected\def\updatettwd{%
  \setbox\@tempboxa\hbox{\normalsize\ttfamily a}
  \ttwd=\wd\@tempboxa
}
\updatettwd
\makeatother
\title{Code documentation to the \pkg{physics2} package}
\author{Zhang Tingxuan}
\date{2023/10/24 v1.0.0}
\begin{document}
  \DocInput{physics2-code.dtx}
\end{document}
%</driver>
% \fi
% 
% \maketitle
% 
% \tableofcontents\newpage
%
%^^A******************************************************************************** [physics2.sty]
% \part{The bare \pkg{physics2}}
% \section{The \pkg{physics2} package}
%    \begin{macrocode}
%<*package>
\NeedsTeXFormat{LaTeX2e}[2020/10/01]
\ProvidesPackage{physics2}
  [2023/10/24 v1.0.0 Tools for typesetting math for physics.]
%    \end{macrocode}
% \subsection{Common variables}
% \begin{variable}{\phy@temp..}
%   \begin{syntax}
%     \cs{phy@temp}\meta{register type}\meta{a|b|c}
%   \end{syntax}
%   Some \LaTeXe\ variables starting with ``|\phy@temp|''. 
%   These variables can be shared by any module of \pkg{physics2}.
% \end{variable}
%    \begin{macrocode}
\newcount  \phy@tempcnta
\newdimen  \phy@tempdima
\newdimen  \phy@tempdimb
\newskip   \phy@tempskipa
\newmuskip \phy@tempmuskipa
\newbox    \phy@tempboxa
\newif     \ifphy@tempswa
\newtoks   \phy@toksa
%    \end{macrocode}
% \subsection{Package requirements and module-loading methods}
% \pkg{physics2} requires \pkg{keyval} (part of the graphics bundle) to 
% process options of modules.
%    \begin{macrocode}
\RequirePackage{keyval}
\def\phy@true{true}
\def\phy@false{false}
%    \end{macrocode}
% \begin{function}{\phy@define@key, \phy@setkeys, \phy@processkeyopt}
%   \begin{syntax}
%     \cs{phy@define@key} \marg{module} \marg{key} \oarg{default value} \marg{code}
%     \cs{phy@setkeys} \marg{module} \marg{key-val list}
%     \cs{phy@processkeyopt} \marg{module}
%   \end{syntax}
%   The position of \cs{phy@processkeyopt} in a \pkg{physics2} module is just the same
%   as the position of \cs{ProcessOptions} in a regular \LaTeX\ package.
% \end{function}
%    \begin{macrocode}
\long\def\phy@define@key#1{\define@key{phy-#1}}
\long\def\phy@setkeys#1{\setkeys{phy-#1}}
\def\phy@processkeyopt#1{\let\reserved@a\@empty%
  \edef\reserved@a{\@ptionlist{\@currname.\@currext}}%
  \edef\reserved@a{\noexpand\phy@setkeys{#1}{\reserved@a}}%
  \reserved@a% the next line thanks to `geometry'
  \AtEndOfPackage{\let\@unprocessedoptions\relax}}
%    \end{macrocode}
% We use almost the same way to load \pkg{physics2} modules as \LaTeXe\ kernel
% does. We use a lot of kernel commands in \LaTeXe.
% 
% \begin{function}{\usephysicsmodule, \phy@requiremodule}
%   \begin{syntax}
%     \cs{usephysicsmodule} \oarg{key-val options} \marg{module} \oarg{key-val options}
%     \cs{phy@requiremodule} \oarg{key-val options} \marg{module} \oarg{key-val options}
%   \end{syntax}
%   \cs{usephysicsmodule} is a user command, and \cs{phy@requiremodule} is a developer
%   command.
% \end{function}
%    \begin{macrocode}
\def\usephysicsmodule{\phy@FWoptions\@pkgextension}
\let\phy@requiremodule\usephysicsmodule
\@onlypreamble\usephysicsmodule
\def\phy@FWoptions#1{\@ifnextchar[%]
  {\phy@FW@ptions#1}{\phy@FW@ptions#1[]}}
\@onlypreamble\phy@FWoptions
\def\phy@FW@ptions#1[#2]#3{\@ifnextchar[%]
  {\phy@FW@pti@ns#1[{#2}]#3}{\phy@FW@pti@ns#1[{#2}]#3[]}}
\@onlypreamble\phy@FW@ptions
\def\phy@FW@pti@ns#1[#2]#3[#4]{%
  \def\reserved@b##1,{%
    \ifx\@nnil##1\relax\else
      \ifx\@nnil##1\@nnil\else
       \noexpand\@onefilewithoptions{phy-##1}[{\unexpanded{#2}}][{#4}]%
       \noexpand\@pkgextension
      \fi
      \expandafter\reserved@b
    \fi}%
    \edef\reserved@a{\zap@space#3 \@empty}%
    \edef\reserved@a{\expandafter\reserved@b\reserved@a,\@nnil,}%
  \reserved@a}
\@onlypreamble\phy@FW@pti@ns
%    \end{macrocode}
% \subsection{The (used to be) \modu{common} module}
% The code below used to be the automatically-loaded \modu{common} module,
% but now we load it together with \pkg{physics2}'s code. This change may
% bring better performance in Windows system.
% 
% Check if \pkg{unicode-math} loaded and (re)define the vert symbols.
% The \cs{relax}'s at the ends of \cs{vert} and \cs{Vert}'s definitions must not be
% removed. They are for \cs{ifx} to compare. \pkg{unicode-math} sets these symbols
% \cs{fam}1, \cs{symoperators} is equal to 1 in \LaTeXe\ kernel. Moreover, we make
% \cs{mid} as a delimiter but it may not work.
%    \begin{macrocode}
\AtBeginDocument{\ifcsname symrm\endcsname
  \protected\def\|{\Udelimiter 0 \symoperators "2016 }%
  \protected\def\vert{\Udelimiter 0 \symoperators "007C\relax}%
  \protected\def\Vert{\Udelimiter 0 \symoperators "2016\relax}%
  \protected\def\mid{\Udelimiter 3 \symoperators "007C }%
\fi}
\protected\def\Vert{\delimiter"026B30D\relax}
\protected\def\mid{\delimiter"326A30C }
%    \end{macrocode}
% \begin{function}{\delopen, \delclose}
%   \begin{syntax}
%     \cs{delopen} \meta{left delimiter}
%     \cs{delclose} \meta{right delimiter}
%   \end{syntax}
%   Actually in \TeX, \cs{left} and \cs{right} will enclose the subformula as ``inner'',
%   but \cs{delopen} and \cs{delclose} will make the subformula an empty open node and
%   a non-empty close node.
% \end{function}
%    \begin{macrocode}
\DeclareRobustCommand\delopen{\mathopen{}\mathclose\bgroup\left}
\DeclareRobustCommand\delclose{\aftergroup\egroup\right}
% Extension to 2e kernel's or amsmath's biggggg commands.
%    \end{macrocode}
% \cs{bBigg@} is a command from \pkg{amsmath}. The code below should
% update with \pkg{amsmath} together.
%    \begin{macrocode}
\ifdefined\bBigg@
  \DeclareRobustCommand\biggg{\bBigg@{3}}
  \DeclareRobustCommand\Biggg{\bBigg@{3.5}}
\else
  \DeclareRobustCommand\biggg[1]{\leavevmode@ifvmode
    {\hbox{$\left#1\vbox to20.5\p@{}\right.\n@space$}}}
  \DeclareRobustCommand\Biggg[1]{\leavevmode@ifvmode
    {\hbox{$\left#1\vbox to23.5\p@{}\right.\n@space$}}}
  \AtBeginDocument{\ifdefined\bBigg@
    \DeclareRobustCommand\biggg{\bBigg@{3}}%
    \DeclareRobustCommand\Biggg{\bBigg@{3.5}}%
  \fi}
\fi
\DeclareRobustCommand\bigggl{\mathopen\biggg}
\DeclareRobustCommand\bigggm{\mathrel\biggg}
\DeclareRobustCommand\bigggr{\mathclose\biggg}
\DeclareRobustCommand\Bigggl{\mathopen\Biggg}
\DeclareRobustCommand\Bigggm{\mathrel\Biggg}
\DeclareRobustCommand\Bigggr{\mathclose\Biggg}
%    \end{macrocode}
% \begin{function}{\phy@mathvphantom}
%   \begin{syntax}
%     \cs{phy@mathvphantom} \marg{math mode material}
%   \end{syntax}
%   This command is just like \cs{vphantom} in \LaTeXe\ kernel but only works
%   in math mode.
% \end{function}
%    \begin{macrocode}
\def\phy@mathvphantom#1{\setbox\phy@tempboxa=\hbox{}%
  \mathchoice
    {\setbox\@tempboxa\hbox{$\displaystyle#1$}%
      \ht\phy@tempboxa=\ht\@tempboxa
      \dp\phy@tempboxa=\dp\@tempboxa
      \box\phy@tempboxa}
    {\setbox\@tempboxa\hbox{$\textstyle#1$}%
      \ht\phy@tempboxa=\ht\@tempboxa
      \dp\phy@tempboxa=\dp\@tempboxa
      \box\phy@tempboxa}
    {\setbox\@tempboxa\hbox{$\scriptstyle#1$}%
      \ht\phy@tempboxa=\ht\@tempboxa
      \dp\phy@tempboxa=\dp\@tempboxa
      \box\phy@tempboxa}
    {\setbox\@tempboxa\hbox{$\scriptscriptstyle#1$}%
      \ht\phy@tempboxa=\ht\@tempboxa
      \dp\phy@tempboxa=\dp\@tempboxa
      \box\phy@tempboxa}%
}
%    \end{macrocode}
% \subsection{The (used to be) \modu{explsetup} module}
% Some common variables and functions for experimental \LaTeX3 syntax.
%    \begin{macrocode}
%<@@=phy>
\ExplSyntaxOn
\int_new:N \l_@@_tmpa_int
\int_new:N \l_@@_tmpb_int
\tl_new:N \l_@@_tmpa_tl
\tl_new:N \l_@@_tmpb_tl
%    \end{macrocode}
% The function that can gobble one token.
%    \begin{macrocode}
\cs_new:Npn \_@@_gobble_i:n #1 { }
\ExplSyntaxOff
%<@@=>
%</package>
%    \end{macrocode}
% 
%^^A******************************************************************************** [l2e modules]
% \part{Modules written in \LaTeXe\ syntax}
% \section{The \modu{ab} module}
% \begingroup\color{gray}%
% \noindent$\langle*\hbox{\sffamily gibberish}\rangle$\par
% This module is important but the code is hard to read. One of the motivations
% I manage \pkg{physics2} with \pkg{DocStrip} is that, when I tried to write a
% new module based on \modu{ab} after 5 months when I maintained \pkg{physics2}
% the last time, I found that I could not understand the code I wrote at all!
% Therefore, it's significant to comment out the alien code in \modu{ab}.\par
% \noindent$\langle/\hbox{\sffamily gibberish}\rangle$\par
% \endgroup
%    \begin{macrocode}
%<*ab>
\ProvidesFile{phy-ab.sty}
  [2023/10/24 `ab' (autobraces) module of physics2]
%    \end{macrocode}
% If you don't know when to use \cs{phy@define@key}, \cs{phy@setkeys} and
% \cs{phy@processkeyopt} in a module, see ahead. In \modu{ab}, the |tightbraces|
% option can control if the automatically-sized braces are tight or not. Do you
% remember \cs{delopen} and \cs{delclose}?
%    \begin{macrocode}
\phy@define@key{ab}{tightbraces}[true]{\def\@phy@abtight{#1}}
\phy@setkeys{ab}{tightbraces=true}
\phy@processkeyopt{ab}
%    \end{macrocode}
% \begin{function}{\phy@abopen, \phy@abclose}
%   \begin{syntax}
%     \cs{phy@abopen} \meta{left delimiter}
%     \cs{phy@abclose} \meta{right delimiter}
%   \end{syntax}
%   They are defined either \{\cs{delopen}, \cs{delclose}\} or
%   \{\cs{left}, \cs{right}\}. If a module requires \modu{ab},
%   these two commands are likely to be used.
% \end{function}
%    \begin{macrocode}
\ifx\@phy@abtight\phy@true
  \let\phy@abopen\delopen
  \let\phy@abclose\delclose
\else
  \let\phy@abopen\left
  \let\phy@abclose\right
\fi
%    \end{macrocode}
% \subsection{The implementation of \cs{ab}}
% This is the alienest part of \modu{ab}. It's better to draw something rather
% than write boring comments. First let's take a look at \cs{ab}'s syntax.
% After \cs{ab} should be a pair of delimiters; take \texttt{()} as an example.
% Between \cs{ab} and ``\texttt{(}'' can be a biggg command or star, or even nothing.
% \cs{ab} is defined as follows:
% \[ \macrodef{ab}{\cs{phy@d@lx} \texttt{\{mb\}} \texttt{\{ab\}}} \]
% where \texttt{ab} is the branch name of |\ab()|, and \texttt{mb} is the branch
% name of |\ab\big()| and |\ab*()|. Then let's see the syntax of \cs{phy@d@lx}.
% \[ \hbox{\ttfamily\cs{phy@d@lx} \marg{biggg or star branch name} \marg{automatic branch name} \{\#3\}} \]
% Here exists an \texttt{\#3}. \texttt{\#3} is one token immediately following \cs{ab},
% which can be \{\,a biggg command or a star\,\} or a ``\texttt{(}'', under our assumption.

% \cs{phy@d@lx} is defined as follows:
% \[
%   \macrodef {phy@d@lx} [\#1: biggg or star branch name, \meta{mb}; \#2: automatic branch name, \meta{ab}; \#3, the token after \cs{ab}]
%     {%
%       \macroifelse {\#3 == biggg or \#3 == star ($\Leftrightarrow$ csname \{phy@del\cs{string}\#3\} is defined)}
%         {let \meta{next cs} = csname \{phy@d@lx\meta{mb}\}}
%         {let \meta{next cs} = csname \{phy@d@lx\meta{ab}\}}
%       \meta{next cs} \#3
%     }
% \]
% The condition should be true when \#3 is |\big| or |*|, and it should be false when
% \#3 is ``|(|''. Accordingly, in math mode,
% \begin{align*}
%   \hbox{\ttfamily\cs{ab} \cs{big} (}&\quad\to\quad\hbox{\ttfamily\cs{phy@d@lxmb} \cs{big} (}\\
%   \hbox{\ttfamily\cs{ab} \phantom{\cs{big}} (}&\quad\to\quad\hbox{\ttfamily\cs{phy@d@lxab} \phantom{\cs{big}} (}
% \end{align*}
% Then we meet two new commands --- \cs{phy@d@lxmb} and \cs{phy@d@lxab}. Syntax is as follows.
% \begin{align*}
%   &\hbox{\ttfamily\cs{phy@d@lxmb} \meta{biggg or *}
%     \meta{left delimiter} \meta{subformula} \meta{right delimiter}}\\
%   &\hbox{\ttfamily\cs{phy@d@lxab} \phantom{\meta{biggg or *}}
%     \meta{left delimiter} \meta{subformula} \meta{right delimiter}}
% \end{align*}
% Notice that \texttt{ab} and \texttt{mb} in the above commands are names of \cs{ab}'s
% two branches --- they are like namespaces. \cs{phy@d@lxmb} and \cs{phy@d@lxab} are
% defined by the following two lines:
% \begin{Verbatim}
% \phy@d@l@genxm{mb}
% \phy@d@l@genxa{ab}
% \end{Verbatim}
% \cs{phy@d@l@genxm} and \cs{phy@d@l@genxa} are defined as follows:
% \begin{align*}
%   & \macrodef {phy@d@l@genxm} [\#1: biggg or star branch name, \meta{mb}]
%     {%
%       \macrodef [12\ttwd] {phy@d@lx\meta{mb}} [\#\#1: biggg or star; \#\#2: left delimiter]
%         {%
%           \cs{begingroup}
%           \macroifelse {\#\#1 == star}
%             {\meta{temp} $\gets$ \cs{relax}}
%             {\meta{temp} $\gets$ \#\#1}
%           csname \{phy@\meta{mb}@\cs{string}\#\#2\} \meta{temp} \#\#2 \macrocr
%           \% requires an \cs{endgroup} after the right delimiter
%         }
%     } \\[1ex]
%     & \macrodef {phy@d@l@genxa} [\#1: automatic branch name, \meta{ab}]
%     {%
%       \macrodef [12\ttwd] {phy@d@lx\meta{ab}} [\#\#1: left delimiter]
%         {%
%           csname \{phy@\meta{ab}@\cs{string}\#\#1\} \#\#1
%         }
%     }
% \end{align*}
% So we can get
% \begin{align*}
%   \hbox{\ttfamily\cs{ab} \cs{big} (}&\quad\to\quad\hbox{\cs{begingroup} csname \{phy@mb@(\} \cs{big}\hskip2\ttwd\ \texttt(}\\
%   \hbox{\ttfamily\cs{ab} *\phantom{big} (}&\quad\to\quad\hbox{\cs{begingroup} csname \{phy@mb@(\} \cs{relax} \texttt(}\\
%   \hbox{\ttfamily\cs{ab} \phantom{\cs{big}} (}&\quad\to\quad\hbox{\hskip11\ttwd\ csname \{phy@ab@(\} \texttt(}
% \end{align*}
% The csnames above (\cs{phy@mb@(} and \cs{phy@ab@(}) are generated with \cs{phy@AB@gen}.
% \[ \hbox{\cs{phy@AB@gen} \marg{branch name} \meta{left delimiter} \marg{arg spec} \marg{definition}} \]
% If \meta{branch name} is \texttt{mb}, \marg{arg spec} should be |mr()|, where |m| is for biggg or star;
% If \meta{branch name} is \texttt{ab}, \marg{arg spec} should be |r()|.
% 
% \note The ``|(|'' in the example above must not be replaced by a subformula braced by a pair of |{}|.
% 
% \begin{function}{\phy@AB@gen}
%   \begin{syntax}
%     \cs{phy@AB@gen} \marg{branch name} \meta{left delimiter} \marg{arg spec} \marg{definition}
%   \end{syntax}
% \end{function}
%    \begin{macrocode}
\def\phy@AB@gen#1#2{\expandafter\DeclareDocumentCommand\csname phy@#1@\string#2\endcsname}
\phy@AB@gen{ab}({r()}{\phy@abopen(#1\phy@abclose)}
\phy@AB@gen{ab}[{r[]}{\phy@abopen[#1\phy@abclose]}
\phy@AB@gen{ab}\{{r\{\}}{\phy@abopen\{#1\phy@abclose\}}
\phy@AB@gen{ab}|{r||}{\phy@abopen|#1\phy@abclose|}
\phy@AB@gen{ab}\|{r\|\|}{\phy@abopen\|#1\phy@abclose\|}
\phy@AB@gen{ab}<{r<>}{\phy@abopen<#1\phy@abclose>}
\phy@AB@gen{ab}\lbrace{r\lbrace\rbrace}{\phy@abopen\lbrace#1\phy@abclose\rbrace}
\phy@AB@gen{ab}\vert{r\vert\vert}{\phy@abopen\vert#1\phy@abclose\vert}
\phy@AB@gen{ab}\Vert{r\Vert\Vert}{\phy@abopen\Vert#1\phy@abclose\Vert}
\phy@AB@gen{ab}\langle{r\langle\rangle}{\phy@abopen\langle#1\phy@abclose\rangle}
%    \end{macrocode}
% \cs{endgroup}'s in the end of the following definitions are corresponding to
% \cs{begingroup}'s in the definition of \cs{phy@d@l@genxm}.
%    \begin{macrocode}
\phy@AB@gen{mb}({mr()}{\mathopen#1(#2\mathclose#1)\endgroup}
\phy@AB@gen{mb}[{mr[]}{\mathopen#1[#2\mathclose#1]\endgroup}
\phy@AB@gen{mb}\{{mr\{\}}{\mathopen#1\lbrace#2\mathclose#1\rbrace\endgroup}
\phy@AB@gen{mb}|{mr||}{\mathopen#1\vert#2\mathclose#1\vert\endgroup}
\phy@AB@gen{mb}\|{mr\|\|}{\mathopen#1\Vert#2\mathclose#1\Vert\endgroup}
\phy@AB@gen{mb}<{mr<>}{\mathopen#1\langle#2\mathclose#1\rangle\endgroup}
\phy@AB@gen{mb}\lbrace{mr\lbrace\rbrace}{\mathopen#1\lbrace#2\mathclose#1\rbrace\endgroup}
\phy@AB@gen{mb}\vert{mr\vert\vert}{\mathopen#1\vert#2\mathclose#1\vert\endgroup}
\phy@AB@gen{mb}\Vert{mr\Vert\Vert}{\mathopen#1\Vert#2\mathclose#1\Vert\endgroup}
\phy@AB@gen{mb}\langle{mr\langle\rangle}{\mathopen#1\langle#2\mathclose#1\rangle\endgroup}
%    \end{macrocode}
% \begin{function}{\phy@del\string.}
%   The syntax seems not important. These following lines seems only for \cs{ifcsname}
%   to judge if the commands are defined.
% \end{function}
%    \begin{macrocode}
\def\phy@del#1#2#3{\phy@abopen#1#3\phy@abclose#2}
\expandafter\def\csname phy@del\string*\endcsname#1#2#3{\mathopen#1#3\mathclose#2}
\expandafter\def\csname phy@del\string\big\endcsname#1#2#3{\bigl#1#3\bigr#2}
\expandafter\def\csname phy@del\string\Big\endcsname#1#2#3{\Bigl#1#3\Bigr#2}
\expandafter\def\csname phy@del\string\bigg\endcsname#1#2#3{\biggl#1#3\biggr#2}
\expandafter\def\csname phy@del\string\Bigg\endcsname#1#2#3{\Biggl#1#3\Biggr#2}
\expandafter\def\csname phy@del\string\biggg\endcsname#1#2#3{\bigggl#1#3\bigggr#2}
\expandafter\def\csname phy@del\string\Biggg\endcsname#1#2#3{\Bigggl#1#3\Bigggr#2}
%    \end{macrocode}
% \begin{function}{\phy@d@lx}
%   \begin{syntax}
%     \cs{phy@d@lx} \marg{biggg or star branch name} \marg{automatic branch name} \{\#3\}
%   \end{syntax}
% \end{function}
%    \begin{macrocode}
\def\phy@d@lx#1#2#3{%
  \ifcsname phy@del\string#3\endcsname
    \def\reserved@a{#1}% #3 is star or \biggg
  \else
    \def\reserved@a{#2}% #3 is delimiter
  \fi
  \csname phy@d@lx\reserved@a\endcsname#3}
%    \end{macrocode}
% \begin{function}{\phy@d@l@genxm, \phy@d@l@genxa}
%   \begin{syntax}
%     \cs{phy@d@l@genxm} \marg{biggg or star branch name}
%     \cs{phy@d@l@genxa} \marg{automatic branch name}
%   \end{syntax}
% \end{function}
%    \begin{macrocode}
\def\phy@d@l@genxm#1{%
  \expandafter\def\csname phy@d@lx#1\endcsname##1##2{%
    \begingroup % \endgroup is at the end of #4 of \phy@AB@gen
    \ifx##1*\let\phy@tempa=\relax\else\let\phy@tempa=##1\fi
    \csname phy@#1@\string##2\endcsname\phy@tempa##2}}
\def\phy@d@l@genxa#1{%
  \expandafter\def\csname phy@d@lx#1\endcsname##1{%
    \csname phy@#1@\string##1\endcsname##1}}
%    \end{macrocode}
% \begin{function}{\phy@d@lxmb, \phy@d@lxab}
%   \begin{syntax}
%     \cs{phy@d@lxmb} \meta{biggg or *} \meta{left delimiter} \meta{subformula} \meta{right delimiter}
%     \cs{phy@d@lxab} \phantom{\meta{biggg or *}} \meta{left delimiter} \meta{subformula} \meta{right delimiter}
%   \end{syntax}
% \end{function}
%    \begin{macrocode}
\phy@d@l@genxm{mb}
\phy@d@l@genxa{ab}
%    \end{macrocode}
% \begin{function}{\ab}
%   The users' command \cs{ab}.
% \end{function}
%    \begin{macrocode}
\DeclareRobustCommand\ab{\phy@d@lx{mb}{ab}}
%    \end{macrocode}
% \subsection{\cs{pab}-like commands}
% This is so simple. No need to comment a lot.
% 
% \begin{function}{\phy@d@l@geny}
%   \begin{syntax}
%     \cs{phy@d@l@geny} \meta{command} \meta{left delimiter} \meta{right delimiter}
%   \end{syntax}
%   This command used to define commands like \cs{pab}.
% \end{function}
%    \begin{macrocode}
\def\phy@d@l@geny#1#2#3{%
  \DeclareDocumentCommand#1{som}{% ##1: star; ##2: bigg (csname); ##3: subformula.
    \IfBooleanTF{##1}%
      {#2##3#3}%
      {\IfValueTF{##2}%
        {\csname##2l\endcsname#2##3\csname##2r\endcsname#3}%
        {\phy@abopen#2##3\phy@abclose#3}%
      }%
  }%
}
\phy@d@l@geny\pab()
\phy@d@l@geny\bab[]
\phy@d@l@geny\Bab\lbrace\rbrace
\phy@d@l@geny\vab\vert\vert
\phy@d@l@geny\aab\langle\rangle
\phy@d@l@geny\Vab\Vert\Vert
%</ab>
%    \end{macrocode}
% \section{The \modu{ab.braket} module}
%    \begin{macrocode}
%<*ab.braket>
\ProvidesFile{phy-ab.braket.sty}
  [2023/10/24 `ab.braket' module of physics2]
%    \end{macrocode}
% This module requires \cs{phy@abopen} and \cs{phy@abclose} from \modu{ab}.
% This module may have conflict with \modu{braket}.
%    \begin{macrocode}
\phy@requiremodule{ab}
\ifdefined\phy@bra@@
  \PackageWarning{physics2}{You cannot load `ab.braket' and `braket'
    modules together.\MessageBreak Only `ab.braket' module works now.}
\fi
%    \end{macrocode}
% \begin{function}{\bra}
%   \begin{syntax}
%     \cs{bra} \char`\<\ \meta{subformula} \char`\|
%   \end{syntax}
% \end{function}
%    \begin{macrocode}
\phy@AB@gen{br.m}<{mr<|}{\mathopen#1\langle#2\mathclose#1\vert\endgroup}
\phy@AB@gen{br.a}<{r<|}{\phy@abopen\langle#1\phy@abclose\vert}
\phy@d@l@genxm{br.m}
\phy@d@l@genxa{br.a}
\DeclareRobustCommand\bra{\phy@d@lx{br.m}{br.a}}
%    \end{macrocode}
% \begin{function}{\ket}
%   \begin{syntax}
%     \cs{ket} \char`\|\ \meta{subformula} \char`\>
%   \end{syntax}
% \end{function}
%    \begin{macrocode}
\phy@AB@gen{kt.m}|{mr|>}{\mathopen#1\vert#2\mathclose#1\rangle\endgroup}
\phy@AB@gen{kt.a}|{r|>}{\phy@abopen\vert#1\phy@abclose\rangle}
\phy@d@l@genxm{kt.m}
\phy@d@l@genxa{kt.a}
\DeclareRobustCommand\ket{\phy@d@lx{kt.m}{kt.a}}
%    \end{macrocode}
% \begin{function}{\braket}
%   \begin{syntax}
%     \cs{braket} \char`\<\ \meta{subformula 1} \char`\|\ \meta{subformula 2} {\color{gray}\textrm{[}\char`\|\ \meta{subformula 3} \textrm{\dots]}} \char`\>
%   \end{syntax}
% \end{function}
%    \begin{macrocode}
\begingroup
\catcode`\|=\active
\gdef\phy@@mb@bk#1#2{\begingroup
  \mathcode`\|="8000\def|{\egroup#1\vert\bgroup}%
  \def\<{\mathrel{<}}\def\>{\mathrel{>}}%
  \mathopen#1\langle\bgroup#2\egroup\mathclose#1\rangle\endgroup}
\gdef\phy@@ab@bk#1{\begingroup
  \mathcode`\|="8000\def|{\egroup\phy@abb@bkv\bgroup}%
  \def\<{\mathrel{<}}\def\>{\mathrel{>}}%
  \phy@abopen\langle\bgroup#1\egroup\phy@abclose\rangle\endgroup}
\endgroup
\def\phy@abb@bkv{\middle\vert}
\phy@AB@gen{bk.m}<{mr<>}{\phy@@mb@bk#1{#2}\endgroup}
\phy@AB@gen{bk.a}<{r<>}{\phy@@ab@bk{#1}}
\phy@d@l@genxm{bk.m}
\phy@d@l@genxa{bk.a}
\DeclareRobustCommand\braket{\phy@d@lx{bk.m}{bk.a}}
%    \end{macrocode}
% \begin{function}{\ketbra}
%   \begin{syntax}
%     \cs{braket} \char`\|\ \meta{subformula 1} \char`\>\ \meta{subformula 2} \char`\<\ \meta{subformula 3} \char`\|
%   \end{syntax}
% \end{function}
%    \begin{macrocode}
\begingroup
\catcode`\<=\active
\catcode`\>=\active
\gdef\phy@@mb@kb#1#2{\begingroup
  \mathcode`\<="8000 \mathcode`\>="8000%
  \def<{#1\langle}\def>{#1\rangle}%
  \def\<{\phy@abb@l}\def\>{\phy@abb@r}%
  \mathopen#1\vert#2\mathclose#1\vert\endgroup}
\endgroup
\gdef\phy@@ab@kb#1>#2<#3\phy@@end{\begingroup
  \def\<{\phy@abb@l}\def\>{\phy@abb@r}%
  \phy@abopen\vert\mathopen{\phy@mathvphantom{#3}}#1\phy@abclose\rangle#2%
  \phy@abopen\langle#3\mathclose{\phy@mathvphantom{#1}}\phy@abclose\vert
\endgroup}
\AtBeginDocument{\ifcsname symbf\endcsname
  \def\phy@abb@l{\Umathchar 3 \symoperators "003C }%
  \def\phy@abb@r{\Umathchar 3 \symoperators "003E }%
\fi}
\def\phy@abb@l{\mathchar"313C }
\def\phy@abb@r{\mathchar"313E }
\phy@AB@gen{kb.m}|{mr||}{\phy@@mb@kb#1{#2}\endgroup}
\phy@AB@gen{kb.a}|{r||}{\phy@@ab@kb#1\phy@@end}
\phy@d@l@genxm{kb.m}
\phy@d@l@genxa{kb.a}
\DeclareRobustCommand\ketbra{\phy@d@lx{kb.m}{kb.a}}
%</ab.braket>
%    \end{macrocode}
% \section{The \modu{braket} module}
%    \begin{macrocode}
%<*braket>
\ProvidesFile{phy-braket.sty}
  [2023/10/24 `braket' module of physics2]
%    \end{macrocode}
% This module requires \cs{phy@abopen} and \cs{phy@abclose} from \modu{ab}.
% This module may have conflict with \modu{ab.braket}.
%    \begin{macrocode}
\phy@requiremodule{ab}
\ifdefined\phy@abb@bkv
  \PackageWarning{physics2}{You cannot load `ab.braket' and `braket'
    modules together.\MessageBreak Only `braket' module works now.}
\fi
%    \end{macrocode}
% \begin{function}{\bra}
%   \begin{syntax}
%     \cs{bra} * \oarg{biggg} \marg{subformula}
%   \end{syntax}
% \end{function}
%    \begin{macrocode}
\DeclareDocumentCommand\bra{ s o m }{%
  \IfBooleanTF{#1}
    {\mathopen\langle#3\mathclose\vert}
    {\IfValueTF{#2}
      {\csname#2l\endcsname\langle#3\csname#2r\endcsname\vert}
      {\phy@abopen\langle#3\phy@abclose\vert}%
    }%
}
%    \end{macrocode}
% \begin{function}{\ket}
%   \begin{syntax}
%     \cs{ket} * \oarg{biggg} \marg{subformula}
%   \end{syntax}
% \end{function}
%    \begin{macrocode}
\DeclareDocumentCommand\ket{ s o m }{%
  \IfBooleanTF{#1}
    {\mathopen\vert#3\mathclose\rangle}
    {\IfValueTF{#2}
      {\csname#2l\endcsname\vert#3\csname#2r\endcsname\rangle}
      {\phy@abopen\vert#3\phy@abclose\rangle}%
    }%
}
%    \end{macrocode}
% \begin{function}{\braket}
%   \begin{syntax}
%     \cs{ket} * [\meta{biggg}, $n \in \{1,2,3\}$] \marg{subformula 1} \dots \marg{subformula $n$}
%   \end{syntax}
% \end{function}
%    \begin{macrocode}
\DeclareDocumentCommand\braket{ s O{} }{%
  \IfBooleanTF{#1}%
    {%
      \gdef\@phy@bk@argnum{ii}%
      \phy@bk@doopt{#2}%
      \gdef\@phy@bk@l{mathopen}%
      \gdef\@phy@bk@m{mathord}%
      \gdef\@phy@bk@r{mathclose}%
    }%
    {%
      \gdef\@phy@bk@argnum{ii}%
      \gdef\@phy@bk@l{phy@abopen}%
      \gdef\@phy@bk@m{middle}%
      \gdef\@phy@bk@r{phy@abclose}%
      \phy@bk@doopt{#2}%
    }%
  \csname phy@bk@in@\@phy@bk@argnum\endcsname%
}
%    \end{macrocode}
% \begin{function}{\phy@bk@in@i, \phy@bk@in@ii, \phy@bk@in@iii}
%   \begin{syntax}
%     \cs{phy@bk@in@\meta{n.roman}} \marg{subformula 1} \dots \marg{subformula $n$}
%   \end{syntax}
%   \meta{n.roman} is $n$ in roman lowercase, where $n \in \{1,2,3\}$.
% \end{function}
%    \begin{macrocode}
\def\phy@bk@in@i#1{%
  \csname\@phy@bk@l\endcsname\langle{#1}%
  \csname\@phy@bk@r\endcsname\rangle}
\def\phy@bk@in@ii#1#2{%
  \csname\@phy@bk@l\endcsname\langle{#1}%
  \csname\@phy@bk@m\endcsname\vert{#2}%
  \csname\@phy@bk@r\endcsname\rangle}
\def\phy@bk@in@iii#1#2#3{%
  \csname\@phy@bk@l\endcsname\langle{#1}%
  \csname\@phy@bk@m\endcsname\vert{#2}%
  \csname\@phy@bk@m\endcsname\vert{#3}%
  \csname\@phy@bk@r\endcsname\rangle}
%    \end{macrocode}
% \begin{function}{\phy@bk@doopt, \phy@bk@do@pt}
%   \begin{syntax}
%     \cs{phy@bk@doopt} \marg{clist}
%   \end{syntax}
%   Parse the optional argument of \cs{braket}. This will add 3 entries to hash.
% \end{function}
%    \begin{macrocode}
\def\@phy@bk@do@pt#1,{\ifx#1\relax\@empty\else
  \edef\reserved@a{\zap@space#1 \@empty}%
  \ifx\reserved@a\@empty\else
    \ifcsname phy@del\expandafter\string\csname\reserved@a\endcsname\endcsname
      \xdef\@phy@bk@l{\reserved@a l}%
      \xdef\@phy@bk@m{\reserved@a}% but not m (m stands for \mathrel)
      \xdef\@phy@bk@r{\reserved@a r}%
    \else
      \ifnum\reserved@a>3%
        \PackageError{physics2}{\string\braket\space can only take 3
          mandatory arguments at most}{Check if you had written a number
          more than 3 in the [optional] argument.}%
      \fi
      \xdef\@phy@bk@argnum{\romannumeral\reserved@a}%
    \fi
  \fi
  \expandafter\@phy@bk@do@pt\fi}
\def\phy@bk@doopt#1{\@phy@bk@do@pt#1,\relax,}
%    \end{macrocode}
% \begin{function}{\ketbra}
%   \begin{syntax}
%     \cs{ketbra} * [\meta{biggg}] \marg{subformula 1} \oarg{between 1 and 2} \marg{subformula 2}
%   \end{syntax}
% \end{function}
%    \begin{macrocode}
\DeclareDocumentCommand\ketbra{ s o m O{} m }{%
  \IfBooleanTF{#1}%
    {\mathopen\vert#3\mathclose\rangle#4\mathopen\langle#5\mathclose\vert}%
    {\IfValueTF{#2}%
      {\csname#2l\endcsname\vert#3\csname#2r\endcsname\rangle#4%
        \csname#2l\endcsname\langle#5\csname#2r\endcsname\vert}%
      {\begingroup
        \phy@abopen\vert\mathopen{\phy@mathvphantom{#5}}#3\phy@abclose\rangle#4%
        \phy@abopen\langle#5\mathclose{\phy@mathvphantom{#3}}\phy@abclose\vert
      \endgroup}%
    }%
}
%</braket>
%    \end{macrocode}
% \section{The \modu{doubleprod} module}
%    \begin{macrocode}
%<*doubleprod>
\ProvidesFile{phy-doubleprod.sty}
  [2023/10/24 `doubleprod' (vertically stacked binary operators) module of physics2]
%    \end{macrocode}
% Boolean options.
%    \begin{macrocode}
\phy@define@key{doubleprod}{crosssymbol}{\def\@phy@dbl@c{#1}}
\phy@define@key{doubleprod}{dotsymbol}{\def\@phy@dbl@d{#1}}
\phy@define@key{doubleprod}{crossscale}{\def\@phy@dbl@sc{#1}}
\phy@define@key{doubleprod}{dotscale}{\def\@phy@dbl@sd{#1}}
\phy@define@key{doubleprod}{crossopenup}{\def\@phy@dbl@oc{#1}}
\phy@define@key{doubleprod}{dotopenup}{\def\@phy@dbl@od{#1}}
\phy@setkeys{doubleprod}{crosssymbol=\times,dotsymbol=\ldotp,
  crossscale=0.8,dotscale=1,crossopenup=.02,dotopenup=.2}
\phy@processkeyopt{doubleprod}
\def\phy@dbl@gen#1#2#3#4{%
  \DeclareRobustCommand#1{\mathbin{\vcenter{\baselineskip\z@skip%
    \lineskip#4\phy@dblcurrf@size%
    \setbox\@tempboxa=\hbox{\fontsize{#2\phy@dblcurrf@size}\z@$#3$}%
    \copy\@tempboxa\box\@tempboxa}}}}
\def\phy@dblcurrf@size{\dimexpr\f@size pt\relax}
\phy@dbl@gen\doublecross\@phy@dbl@sc\@phy@dbl@c\@phy@dbl@oc
\phy@dbl@gen\doubledot\@phy@dbl@sd\@phy@dbl@d\@phy@dbl@od
%</doubleprod>
%    \end{macrocode}
%^^A******************************************************************************** [l3 modules]
% \part{Modules written in \LaTeX3 syntax}
% We use |phy| as the namespace for \pkg{physics2} modules.
%    \begin{macrocode}
%<@@=phy>
%    \end{macrocode}
% \section{The \modu{diagmat} module}
%    \begin{macrocode}
%<*diagmat>
\ProvidesExplFile{phy-diagmat.sty}{2023/10/24}{}
  {`diagmat' module of physics2}
\RequirePackage { amsmath }
\phy@define@key { diagmat } { empty } [ 0 ] { \tl_gset:Nn \l_@@_mat_empty_tl { #1 } }
%    \end{macrocode}
% This module requires some new variables.
%    \begin{macrocode}
\clist_new:N \l_@@_mat_diag_clist
\clist_new:N \l_@@_mat_tmpa_clist
\clist_new:N \l_@@_mat_tmpb_clist
\clist_new:N \l_@@_mat_diagpos_clist
\int_new:N \l_@@_mat_dim_int
\tl_new:N \l_@@_mat_line_tl
\tl_new:N \l_@@_diagmat_tl
\tl_new:N \l_@@_mat_empty_tl
\tl_gset:Nn \l_@@_mat_empty_tl { 0 }
\phy@processkeyopt { diagmat }
\keys_define:nn { phy/diagmat }
  {
    empty .tl_set:N = \l_@@_mat_empty_tl ,
  }
%    \end{macrocode}
% \begin{function}{\diagmat}
%   \begin{syntax}
%     \cs{\meta{delimiter type}diagmat} \oarg{key-val list} \marg{diagonal}
%   \end{syntax}
% \end{function}
%    \begin{macrocode}
\DeclareDocumentCommand \diagmat { O{} m }
  { \_@@_diagmat_type:nnn { } { #1 } { #2 } }
\DeclareDocumentCommand \pdiagmat { O{} m }
  { \_@@_diagmat_type:nnn { p } { #1 } { #2 } }
\DeclareDocumentCommand \bdiagmat { O{} m }
  { \_@@_diagmat_type:nnn { b } { #1 } { #2 } }
\DeclareDocumentCommand \Bdiagmat { O{} m }
  { \_@@_diagmat_type:nnn { B } { #1 } { #2 } }
\DeclareDocumentCommand \vdiagmat { O{} m }
  { \_@@_diagmat_type:nnn { v } { #1 } { #2 } }
\DeclareDocumentCommand \Vdiagmat { O{} m }
  { \_@@_diagmat_type:nnn { V } { #1 } { #2 } }
%    \end{macrocode}
% \begin{function}{\_@@_diagmat_type:nnn}
%   \begin{syntax}
%     \cs{_@@_diagmat_type:nnn} \marg{delimiter type} \marg{key-val list} \marg{diagonal}
%   \end{syntax}
% \end{function}
%    \begin{macrocode}
\cs_new:Npn \_@@_diagmat_type:nnn #1#2#3
  {
    \group_begin:
    \clist_set:Nn \l_@@_mat_diag_clist { #3 }
    \int_set:Nn \l_@@_mat_dim_int { \clist_count:N \l_@@_mat_diag_clist }
    \int_compare:nNnT { \l_@@_mat_dim_int } > { \value { MaxMatrixCols } }
      { \setcounter { MaxMatrixCols } { \l_@@_mat_dim_int } }
    \keys_set:nn { phy/diagmat } { #2 }
    \tl_gclear:N \l_@@_diagmat_tl
    \int_step_inline:nnn { 0 } { \l_@@_mat_dim_int - 1 }
      {
        \int_step_inline:nnn { 0 } { \l_@@_mat_dim_int - 1 }
          {
            \int_compare:nNnTF { ##1 } = { ####1 }
              {
                \clist_gpop:NN \l_@@_mat_diag_clist \l_@@_tmpa_tl
                \tl_if_empty:NTF \l_@@_tmpa_tl
                  { \tl_gput_right:Nn \l_@@_mat_line_tl { \l_@@_mat_empty_tl } }
                  { \tl_gput_right:Nx \l_@@_mat_line_tl { \l_@@_tmpa_tl } }
              }
              { \tl_gput_right:Nn \l_@@_mat_line_tl { \l_@@_mat_empty_tl } }
%    \end{macrocode}
% Add |&| (or |\\|) between matrix entries.
%    \begin{macrocode}
            \int_compare:nNnTF { ####1 } = { \l_@@_mat_dim_int - 1 }
              {
                \tl_gput_right:Nn \l_@@_mat_line_tl { \\ }
              }
              {
                \tl_gput_right:Nn \l_@@_mat_line_tl { & }
              }
          }
        \tl_gput_right:Nx \l_@@_diagmat_tl { \l_@@_mat_line_tl }
        \tl_gclear:N \l_@@_mat_line_tl
      }
    \begin { #1 matrix }
      \tl_use:N \l_@@_diagmat_tl
    \end { #1 matrix }
    \group_end:
  }
%</diagmat>
%    \end{macrocode}
% \section{The \modu{xmat} module}
%    \begin{macrocode}
%<*xmat>
\ProvidesExplFile{phy-xmat.sty}{2023/10/24}{}
  {`xmat' module of physics2}
\RequirePackage { amsmath }
\phy@define@key { xmat } { showtop }
  { \int_gset:Nn \l_@@_xmat_showtop_int { #1 } }
\phy@define@key { xmat } { showleft }
  { \int_gset:Nn \l_@@_xmat_showleft_int { #1 } }
%    \end{macrocode}
% This module requires some new variables.
%    \begin{macrocode}
\bool_new:N \l_@@_xmat_extra_vdots_bool
\bool_new:N \l_@@_xmat_extra_cdots_bool
\int_new:N \l_@@_xmat_showtop_int
\int_new:N \l_@@_xmat_showleft_int
\tl_new:N \l_@@_xmat_tl
\int_gset:Nn \l_@@_xmat_showtop_int { \value { MaxMatrixCols } - 2 }
\int_gset:Nn \l_@@_xmat_showleft_int { \value { MaxMatrixCols } - 2 }
\cs_new:Npn \_@@_xmat_entry_format:nnn #1#2#3
  {
    #1 \c_math_subscript_token { #2 #3 }
  }
\phy@processkeyopt { xmat }
\DeclareDocumentCommand \xmat { O{} m m m }
 { \_@@_xmat_type:nnnnn { } { #1 } { #2 } { #3 } { #4 } }
\DeclareDocumentCommand \pxmat { O{} m m m }
 { \_@@_xmat_type:nnnnn { p } { #1 } { #2 } { #3 } { #4 } }
\DeclareDocumentCommand \bxmat { O{} m m m }
 { \_@@_xmat_type:nnnnn { b } { #1 } { #2 } { #3 } { #4 } }
\DeclareDocumentCommand \Bxmat { O{} m m m }
 { \_@@_xmat_type:nnnnn { B } { #1 } { #2 } { #3 } { #4 } }
\DeclareDocumentCommand \vxmat { O{} m m m }
 { \_@@_xmat_type:nnnnn { v } { #1 } { #2 } { #3 } { #4 } }
\DeclareDocumentCommand \Vxmat { O{} m m m }
 { \_@@_xmat_type:nnnnn { V } { #1 } { #2 } { #3 } { #4 } }
\keys_define:nn { phy/xmat }
  {
    format  .cs_set:Np = \_@@_xmat_entry_format:nnn #1#2#3 ,
    showtop .int_set:N = \l_@@_xmat_showtop_int ,
    showleft.int_set:N = \l_@@_xmat_showleft_int ,
  }
%    \end{macrocode}
% \begin{function}[pTF]{\_@@_if_digits_only:n}
%   \begin{syntax}
%     \cs{_@@_if_digits_only:nTF} \marg{token list} \marg{true code} \marg{flase code}
%   \end{syntax}
%   Use \LaTeX3 regular expression to tell if \meta{token list}
%   (the numbers of rows or columns) contain digits only. 
% \end{function}
%    \begin{macrocode}
\prg_new_conditional:Npnn \_@@_if_digits_only:n #1 { TF }
  {
    \regex_match:nnTF { \A [[:digit:]]* \Z } { #1 }
      { \prg_return_true: } { \prg_return_false: }
  }
%    \end{macrocode}
% \begin{function}{\_@@_xmat_type:nnnnn}
%   \begin{syntax}
%     \cs{_@@_xmat_type:nnnnn} \marg{delimiter type} \marg{key-val list} \marg{common entry} \marg{row number} \marg{column number}
%   \end{syntax}
% \end{function}
%    \begin{macrocode}
\cs_new:Npn \_@@_xmat_type:nnnnn #1#2#3#4#5
  {
    \group_begin:
    \tl_gclear:N \l_@@_xmat_tl
    \keys_set:nn { phy/xmat } { #2 } % 
    \_@@_if_digits_only:nTF { #4 }
      {
        \int_compare:nNnTF { #4 } < { \l_@@_xmat_showtop_int + 1 }
          {
            \int_set:Nn \l_@@_xmat_showtop_int { #4 }
            \bool_set_false:N \l_@@_xmat_extra_vdots_bool
          }
          {
            \bool_set_true:N \l_@@_xmat_extra_vdots_bool
          }
      }
      {
        \bool_set_true:N \l_@@_xmat_extra_vdots_bool
      }
    \_@@_if_digits_only:nTF { #5 }
      {
        \int_compare:nNnTF { #5 } < { \l_@@_xmat_showleft_int + 1 }
          {
            \int_set:Nn \l_@@_xmat_showleft_int { #5 }
            \bool_set_false:N \l_@@_xmat_extra_cdots_bool
          }
          {
            \bool_set_true:N \l_@@_xmat_extra_cdots_bool
          }
      }
      {
        \bool_set_true:N \l_@@_xmat_extra_cdots_bool
      }
    \int_step_inline:nn { \l_@@_xmat_showtop_int }
      {
        \tl_put_right:Nn \l_@@_xmat_tl
          { \_@@_xmat_entry_format:nnn { #3 } { ##1 } { 1 } }
        \int_step_inline:nnn { 2 } { \l_@@_xmat_showleft_int }
          {
            \tl_put_right:Nn \l_@@_xmat_tl
              { & \_@@_xmat_entry_format:nnn { #3 } { ##1 } { ####1 } }
          }
        \bool_if:NT \l_@@_xmat_extra_cdots_bool
          {
            \tl_put_right:Nn \l_@@_xmat_tl
              { & \cdots & \_@@_xmat_entry_format:nnn { #3 } { ##1 } { #5 } }
          }
        \tl_put_right:Nn \l_@@_xmat_tl { \\ }
      }
    \bool_if:NT \l_@@_xmat_extra_vdots_bool
      {
        \tl_put_right:Nn \l_@@_xmat_tl { \vdots }
        \prg_replicate:nn { \l_@@_xmat_showleft_int - 1 }
          {
            \tl_put_right:Nn \l_@@_xmat_tl { & \vdots }
          }
        % Add \ddots if vdots_bool and cdots_bool be true simultaneously.
        \bool_if:NT \l_@@_xmat_extra_cdots_bool
          {
            \tl_put_right:Nn \l_@@_xmat_tl { & \ddots & \vdots }
          } % else relax
        \tl_put_right:Nn \l_@@_xmat_tl { \\ }
        % The last row.
        \tl_put_right:Nn \l_@@_xmat_tl
          { \_@@_xmat_entry_format:nnn { #3 } { #4 } { 1 } }
        \int_step_inline:nnn { 2 } { \l_@@_xmat_showleft_int }
          {
            \tl_put_right:Nn \l_@@_xmat_tl
              { & \_@@_xmat_entry_format:nnn { #3 } { #4 } { ##1 } }
          }
        \bool_if:NT \l_@@_xmat_extra_cdots_bool
          {
            \tl_put_right:Nn \l_@@_xmat_tl
              { & \cdots & \_@@_xmat_entry_format:nnn { #3 } { #4 } { #5 } }
          }
      } % else relax
    \begin { #1 matrix }
      \tl_use:N \l_@@_xmat_tl
    \end { #1 matrix }
    \group_end:
  }
%</xmat>
%    \end{macrocode}
% This part ends here.
%    \begin{macrocode}
%<@@=>
%    \end{macrocode}
%^^A******************************************************************************** [l2e legacy]
% \part{Legacy modules written in \LaTeXe\ syntax}
% \section{The \modu{ab.legacy} module}
%    \begin{macrocode}
%<*ab.legacy>
\ProvidesFile{phy-ab.legacy.sty}
  [2023/10/24 `ab.legacy' module of physics2]
%    \end{macrocode}
% Requires \modu{ab}'s |tight| option.
%    \begin{macrocode}
\phy@requiremodule{ab}
\phy@define@key{ab.legacy}{order}[\mathcal{O}]{\def\phy@ab@ordersym{#1}}
\phy@setkeys{ab.legacy}{order}
\phy@processkeyopt{ab.legacy}
\phy@d@l@geny\abs\vert\vert
\phy@d@l@geny\norm\Vert\Vert
\DeclareDocumentCommand\order{som}{%
  \phy@ab@ordersym
  \IfBooleanTF{#1}
    {(#3)}
    {\IfValueTF{#2}
      {\csname#2l\endcsname(#3\csname#2r\endcsname)}
      {\phy@abopen(#3\phy@abclose)}%
    }%
}
\phy@d@l@geny\eval.\vert
\phy@d@l@geny\peval(\vert
\phy@d@l@geny\beval[\vert
%</ab.legacy>
%    \end{macrocode}
% \section{The \modu{nabla.legacy} module}
%    \begin{macrocode}
%<*nabla.legacy>
\ProvidesFile{phy-nabla.legacy.sty}
  [2023/10/24 `nabla.legacy' module of physics2]
\phy@requiremodule{ab}
%    \end{macrocode}
% Requires \pkg{fixdif} version 2.$x$.
%    \begin{macrocode}
\RequirePackage{fixdif}[2023/01/31]
\letdif\phy@nl@nabla{nabla}
\AtBeginDocument{\ifcsname div\endcsname\let\divsymbol\div\fi
  \DeclareRobustCommand\grad{\phy@nl@nabla\ab}%
  \DeclareRobustCommand\div{\phy@nl@nabla\cdot\ab}%
  \DeclareRobustCommand\curl{\phy@nl@nabla\times\ab}%
  \DeclareRobustCommand\laplacian{\phy@nl@nabla^2\ab}%
}
%</nabla.legacy>
%    \end{macrocode}
% \section{The \modu{op.legacy} module}
%    \begin{macrocode}
%<*op.legacy>
\ProvidesFile{phy-op.legacy.sty}
  [2023/10/24 `op.legacy' module of physics2]
\phy@define@key{op.lega}{ReIm}[true]{\def\phy@reserveda{#1}}
\phy@define@key{op.lega}{PV}{\def\@phy@oplega@PV{#1}}
\phy@define@key{op.lega}{pv}{\def\@phy@oplega@pv{#1}}
\phy@setkeys{op.lega}{PV=\mathcal{P},pv={p.v.},ReIm=true}
\phy@processkeyopt{ab}
\DeclareRobustCommand\asin{\mathop{\operator@font asin}\nolimits}
\DeclareRobustCommand\acos{\mathop{\operator@font acos}\nolimits}
\DeclareRobustCommand\atan{\mathop{\operator@font atan}\nolimits}
\DeclareRobustCommand\acsc{\mathop{\operator@font acsc}\nolimits}
\DeclareRobustCommand\asec{\mathop{\operator@font asec}\nolimits}
\DeclareRobustCommand\acot{\mathop{\operator@font acot}\nolimits}
\DeclareRobustCommand\Tr{\mathop{\operator@font Tr}\nolimits}
\DeclareRobustCommand\tr{\mathop{\operator@font tr}\nolimits}
\DeclareRobustCommand\rank{\mathop{\operator@font rank}\nolimits}
\DeclareRobustCommand\erf{\mathop{\operator@font erf}\nolimits}
\DeclareRobustCommand\Res{\mathop{\operator@font Res}\nolimits}
\DeclareRobustCommand\res{\mathop{\operator@font res}\nolimits}
\DeclareRobustCommand\PV{\mathord{\@phy@oplega@PV}}
\DeclareRobustCommand\pv{\mathop{\operator@font\@phy@oplega@pv{}}\nolimits}
%    \end{macrocode}
% Restore \cs{Re} and \cs{Im} in \cs{Resymbol} and \cs{Imsymbol}. The \cs{AtBeginDocument}
% hook is used for the compatibility of \pkg{unicode-math}.
%    \begin{macrocode}
\ifx\phy@reserveda\phy@true
\AtBeginDocument{%
  \let\Resymbol\Re%
  \let\Imsymbol\Im%
  \DeclareRobustCommand\Re{\mathop{\operator@font Re}\nolimits}%
  \DeclareRobustCommand\Im{\mathop{\operator@font Im}\nolimits}%
}
\fi
%</op.legacy>
%    \end{macrocode}
% \section{The \modu{qtext.legacy} module}
% This module is written for the compatibility with the bad commands provided by
% \pkg{physics} only. The commands in this module should NEVER be used!
%    \begin{macrocode}
%<*qtext.legacy>
\ProvidesFile{phy-qtext.legacy.sty}
  [2023/10/24 `qtext.legacy' module of physics2.sty]
\RequirePackage{amstext}
\def\phy@qtext@#1#2{#1\text{#2}\quad}
\DeclareRobustCommand\qqtext{\@ifstar{\phy@qtext@{}}{\phy@qtext@\quad}}
\DeclareRobustCommand\qq{\qqtext}
\DeclareRobustCommand\qcomma{,\quad}
\DeclareRobustCommand\qc{\qcomma}
\DeclareRobustCommand\qcc{\@ifstar{\phy@qtext@{}{c.c}}{\phy@qtext@\quad{c.c}}}
\def\phy@qtext@lega@gen@#1{%
  \expandafter\DeclareRobustCommand\csname q#1\endcsname%
    {\@ifstar{\phy@qtext@{}{#1}}{\phy@qtext@\quad{#1}}}}
\phy@qtext@lega@gen@{if}
\phy@qtext@lega@gen@{then}
\phy@qtext@lega@gen@{else}
\phy@qtext@lega@gen@{otherwise}
\phy@qtext@lega@gen@{unless}
\phy@qtext@lega@gen@{give}
\phy@qtext@lega@gen@{using}
\phy@qtext@lega@gen@{unless}
\phy@qtext@lega@gen@{assume}
\phy@qtext@lega@gen@{since}
\phy@qtext@lega@gen@{let}
\phy@qtext@lega@gen@{for}
\phy@qtext@lega@gen@{all}
\phy@qtext@lega@gen@{even}
\phy@qtext@lega@gen@{odd}
\phy@qtext@lega@gen@{integer}
\phy@qtext@lega@gen@{and}
\phy@qtext@lega@gen@{or}
\phy@qtext@lega@gen@{as}
\phy@qtext@lega@gen@{in}
%</qtext.legacy>
%    \end{macrocode}
%^^A******************************************************************************** [l3 legacy]
% \part{Legacy modules written in \LaTeX3 syntax}
%    \begin{macrocode}
%<@@=phy>
%    \end{macrocode}
% \section{The \modu{bm-um.legacy} module}
%    \begin{macrocode}
%<*bm-um.legacy>
\ProvidesExplFile{phy-bm-um.legacy.sty}{2023/10/24}{}
  {`bm-um.legacy' module of physics2}
\AtBeginDocument
  {
    \cs_if_exist:cF { symbf }
      {
        \PackageError { physics2 }
          {
            The ~ `bm-um.legacy' ~ module ~ requires ~
            `unicode-math' ~ package
          }
          {
            Have ~ you ~ used ~ `unicode-math' ~
            in ~ the ~ preamble?
          }
      }
  }
\DeclareDocumentCommand \bm { m }
  {
    \mode_if_math:TF
      {
        \tl_if_head_eq_catcode:nNTF { #1 } A
          {
            \symbfit { #1 }
          }
          {
            \symbf { #1 }
          }
      }
      {
        \PackageError { physics2 }
          {
            The ~ \string\bm\space command ~ should ~ be ~
            used ~ in ~ math ~ mode ~ only. \MessageBreak
            This ~ is ~ an ~ error ~ from ~ `bm-um.legacy' ~ module
          }
          {
            Check ~ if ~ any ~ `\string\bm' ~ is ~ out ~
            of ~ math ~ mode.
          }
      }
  }
%</bm-um.legacy>
%    \end{macrocode}
% This part ends here.
%    \begin{macrocode}
%<@@=>
%    \end{macrocode}
\endinput