% \DeclareSourceFile{ab}
%
%
%
% \section{The \modu{ab} module}
% This module is important but the code is hard to read. One of the motivations
% I manage \pkg{physics2} with \pkg{DocStrip} is that, when I tried to write a
% new module based on \modu{ab} after 5 months when I maintained \pkg{physics2}
% the last time, I found that I could not understand the code I wrote at all!
% Therefore, it's significant to comment out the alien code in \modu{ab}.
%    \begin{macrocode}
%<*ab>
\ProvidesFile{phy-ab.sty}
  [2024/06/06 `ab' (autobraces) module of physics2]
%    \end{macrocode}
% If you don't know when to use \cs{phy@define@key}, \cs{phy@setkeys} and
% \cs{phy@processkeyopt} in a module, see ahead. In \modu{ab}, the |tightbraces|
% option can control if the automatically-sized braces are tight or not. Do you
% remember \cs{delopen} and \cs{delclose}?
%    \begin{macrocode}
\phy@define@key{ab}{tightbraces}[true]{\def\@phy@abtight{#1}}
\phy@setkeys{ab}{tightbraces=true}
\phy@processkeyopt{ab}
%    \end{macrocode}
% \begin{function}{\phy@abopen, \phy@abclose}
%   \begin{syntax}
%     \cs{phy@abopen} \meta{left delimiter}
%     \cs{phy@abclose} \meta{right delimiter}
%   \end{syntax}
%   They are defined either \{\cs{delopen}, \cs{delclose}\} or
%   \{\cs{left}, \cs{right}\}. If a module requires \modu{ab},
%   these two commands are likely to be used.
% \end{function}
%    \begin{macrocode}
\ifx\@phy@abtight\phy@true
  \let\phy@abopen\delopen
  \let\phy@abclose\delclose
\else
  \let\phy@abopen\left
  \let\phy@abclose\right
\fi
%    \end{macrocode}
% \subsection{The implementation of \cs{ab}}
% This is the alienest part of \modu{ab}. It's better to draw something rather
% than write boring comments. Let's take a look at \cs{ab}'s syntax.
% After \cs{ab} should be a pair of delimiters; take \texttt{()} as an example.
% Between \cs{ab} and ``\texttt{(}'' can be a biggg command or star, or even nothing.
% \cs{ab} is defined as follows:
% \[ \macrodef{ab}{\cs{phy@d@lx} \texttt{\{mb\}} \texttt{\{ab\}}} \]
% where \texttt{ab} is the branch name of |\ab()|, and \texttt{mb} is the branch
% name of |\ab\big()| and |\ab*()|. Then let's see the syntax of \cs{phy@d@lx}.
% \[ \hbox{\ttfamily\cs{phy@d@lx} \marg{star or biggg branch name} \marg{automatic branch name} \meta{tok}} \]
% Here exists \meta{tok} as \texttt{\#3}. It is one token immediately following \cs{ab},
% which can be \{\,a biggg command or a star\,\} or a ``\texttt{(}'', under our assumption.
% 
% \cs{phy@d@lx} is defined as follows:
% \[
%   \macrodef {phy@d@lx} [\#1: star or biggg branch name, \meta{mb}; \#2: automatic branch name, \meta{ab}; \#3: the token after \cs{ab}, \meta{tok}]
%     {%
%       \macroifelse {\meta{tok} == biggg or \meta{tok} == star}
%         {let \meta{next cs} = csname \{phy@d@lx\meta{mb}\}}
%         {let \meta{next cs} = csname \{phy@d@lx\meta{ab}\}}
%       \meta{next cs} \meta{tok}
%     }
% \]
% The condition should be true when \meta{tok} is |\big| or |*|, and be false when
% \meta{tok} is ``|(|''. Accordingly,
% \begin{align*}
%   \hbox{\ttfamily\cs{ab} \cs{big} (}&\quad\to\quad\hbox{\ttfamily\cs{phy@d@lxmb} \cs{big} (}\\
%   \hbox{\ttfamily\cs{ab} \phantom{\cs{big}} (}&\quad\to\quad\hbox{\ttfamily\cs{phy@d@lxab} \phantom{\cs{big}} (}
% \end{align*}
% Then we meet two new commands --- \cs{phy@d@lxmb} and \cs{phy@d@lxab}. Syntax is as follows.
% \begin{align*}
%   &\hbox{\ttfamily\cs{phy@d@lxmb} \meta{biggg or *}
%     \meta{left delimiter} \meta{subformula} \meta{right delimiter}}\\
%   &\hbox{\ttfamily\cs{phy@d@lxab} \phantom{\meta{biggg or *}}
%     \meta{left delimiter} \meta{subformula} \meta{right delimiter}}
% \end{align*}
% Notice that \texttt{ab} and \texttt{mb} in the above commands are names of \cs{ab}'s
% two branches --- they are like namespaces. \cs{phy@d@lxmb} and \cs{phy@d@lxab} are
% defined by the following two lines:
% \begin{Verbatim}
% \phy@d@l@genxm{mb}
% \phy@d@l@genxa{ab}
% \end{Verbatim}
% \cs{phy@d@l@genxm} and \cs{phy@d@l@genxa} are defined as follows:
% \begin{align*}
%   & \macrodef {phy@d@l@genxm} [\#1: biggg or star branch name, \meta{mb}]
%     {%
%       \macrodef [12\ttwd] {phy@d@lx\meta{mb}} [\#\#1: biggg or star; \#\#2: left delimiter]
%         {%
%           \cs{begingroup}
%           \macroifelse {\#\#1 == star}
%             {\meta{temp} $\gets$ \cs{relax}}
%             {\meta{temp} $\gets$ \#\#1}
%           csname \{phy@\meta{mb}@\cs{string}\#\#2\} \meta{temp} \#\#2 \macrocr
%           \% requires an \cs{endgroup} after the right delimiter
%         }
%     } \\[1ex]
%     & \macrodef {phy@d@l@genxa} [\#1: automatic branch name, \meta{ab}]
%     {%
%       \macrodef [12\ttwd] {phy@d@lx\meta{ab}} [\#\#1: left delimiter]
%         {%
%           csname \{phy@\meta{ab}@\cs{string}\#\#1\} \#\#1
%         }
%     }
% \end{align*}
% So we can get
% \begin{align*}
%   \hbox{\ttfamily\cs{ab} \cs{big} (}&\quad\to\quad\hbox{\cs{begingroup} csname \{phy@mb@(\} \cs{big}\hskip2\ttwd\ \texttt(}\\
%   \hbox{\ttfamily\cs{ab} *\phantom{big} (}&\quad\to\quad\hbox{\cs{begingroup} csname \{phy@mb@(\} \cs{relax} \texttt(}\\
%   \hbox{\ttfamily\cs{ab} \phantom{\cs{big}} (}&\quad\to\quad\hbox{\hskip11\ttwd\ csname \{phy@ab@(\} \texttt(}
% \end{align*}
% The csnames above (\cs{phy@mb@(} and \cs{phy@ab@(}) are generated with \cs{phy@AB@gen}.
% \[ \hbox{\cs{phy@AB@gen} \marg{branch name} \meta{left delimiter} \marg{arg spec} \marg{definition}} \]
% If \meta{branch name} is \texttt{mb}, \marg{arg spec} should be |mr()|, where |m| is for biggg or star;
% If \meta{branch name} is \texttt{ab}, \marg{arg spec} should be |r()|.
% 
% \note The ``|(|'' in the example above must not be replaced by a subformula braced by a pair of |{}|.
% 
% \begin{function}{\phy@AB@gen}
%   \begin{syntax}
%     \cs{phy@AB@gen} \marg{branch name} \meta{left delimiter} \marg{arg spec} \marg{definition}
%   \end{syntax}
% \end{function}
%    \begin{macrocode}
\def\phy@AB@gen#1#2{\expandafter\DeclareDocumentCommand\csname phy@#1@\string#2\endcsname}
\phy@AB@gen{ab}({+r()}{\phy@abopen(#1\phy@abclose)}
\phy@AB@gen{ab}[{+r[]}{\phy@abopen[#1\phy@abclose]}
\phy@AB@gen{ab}\{{+r\{\}}{\phy@abopen\{#1\phy@abclose\}}
\phy@AB@gen{ab}|{+r||}{\phy@abopen|#1\phy@abclose|}
\phy@AB@gen{ab}\|{+r\|\|}{\phy@abopen\|#1\phy@abclose\|}
\phy@AB@gen{ab}<{+r<>}{\phy@abopen<#1\phy@abclose>}
\phy@AB@gen{ab}\lbrace{+r\lbrace\rbrace}{\phy@abopen\lbrace#1\phy@abclose\rbrace}
\phy@AB@gen{ab}\vert{+r\vert\vert}{\phy@abopen\vert#1\phy@abclose\vert}
\phy@AB@gen{ab}\Vert{+r\Vert\Vert}{\phy@abopen\Vert#1\phy@abclose\Vert}
\phy@AB@gen{ab}\langle{+r\langle\rangle}{\phy@abopen\langle#1\phy@abclose\rangle}
%    \end{macrocode}
% \cs{endgroup}'s in the end of the following definitions are corresponding to
% \cs{begingroup}'s in the definition of \cs{phy@d@l@genxm}.
%    \begin{macrocode}
\phy@AB@gen{mb}({m+r()}{\mathopen#1(#2\mathclose#1)\endgroup}
\phy@AB@gen{mb}[{m+r[]}{\mathopen#1[#2\mathclose#1]\endgroup}
\phy@AB@gen{mb}\{{m+r\{\}}{\mathopen#1\lbrace#2\mathclose#1\rbrace\endgroup}
\phy@AB@gen{mb}|{m+r||}{\mathopen#1\vert#2\mathclose#1\vert\endgroup}
\phy@AB@gen{mb}\|{m+r\|\|}{\mathopen#1\Vert#2\mathclose#1\Vert\endgroup}
\phy@AB@gen{mb}<{m+r<>}{\mathopen#1\langle#2\mathclose#1\rangle\endgroup}
\phy@AB@gen{mb}\lbrace{m+r\lbrace\rbrace}{\mathopen#1\lbrace#2\mathclose#1\rbrace\endgroup}
\phy@AB@gen{mb}\vert{m+r\vert\vert}{\mathopen#1\vert#2\mathclose#1\vert\endgroup}
\phy@AB@gen{mb}\Vert{m+r\Vert\Vert}{\mathopen#1\Vert#2\mathclose#1\Vert\endgroup}
\phy@AB@gen{mb}\langle{m+r\langle\rangle}{\mathopen#1\langle#2\mathclose#1\rangle\endgroup}
%    \end{macrocode}
% \begin{function}{\phy@ab@chk\biggg,\phy@ab@chk*}
%   \begin{syntax}
%     \cs{ifcsname} phy@ab@chk\cs{string}\meta{token to be tested}\cs{endcsname}
%   \end{syntax}
%   This is expanded to true only if \meta{token to be tested} is \texttt{*} or a command
%   like \cs{biggg}. This is related to \cs{DeclareBiggg} command in \texttt{physics2.sty}.
% \end{function}
% \begin{function}{\phy@d@lx}
%   \begin{syntax}
%     \cs{phy@d@lx} \marg{biggg or star branch name} \marg{automatic branch name} \meta{next token}
%   \end{syntax}
%   When \meta{next token} is a \texttt{*} or command like \cs{biggg}, let \cs{reserved@a} be 
%   \meta{biggg or star branch name}, that is, \texttt{mb}; when \meta{next token} is a left delimiter,
%   let \cs{reserved@a} be \meta{automatic branch name}, that is, \texttt{ab}. Finally this command is
%   expanded to \texttt{\cs{csname} phy@d@lx\meta{branch name}\cs{endcsname} \meta{next token}}.
% \end{function}
%    \begin{macrocode}
\long\def\phy@d@lx#1#2#3{%
  \ifcsname phy@ab@chk\string#3\endcsname\def\reserved@a{#1}%
  \else\def\reserved@a{#2}%
  \fi\csname phy@d@lx\reserved@a\endcsname#3}
%    \end{macrocode}
% \begin{function}{\phy@d@l@genxm, \phy@d@l@genxa}
%   \begin{syntax}
%     \cs{phy@d@l@genxm} \marg{biggg or star branch name}
%     \cs{phy@d@l@genxa} \marg{automatic branch name}
%   \end{syntax}
% \end{function}
%    \begin{macrocode}
\def\phy@d@l@genxm#1{%
  \long\expandafter\def\csname phy@d@lx#1\endcsname##1##2{%
    \begingroup % \endgroup is at the end of #4 of \phy@AB@gen
    \ifx##1*\let\phy@tempa=\relax\else\let\phy@tempa=##1\fi
    \csname phy@#1@\string##2\endcsname\phy@tempa##2}}
\def\phy@d@l@genxa#1{%
  \long\expandafter\def\csname phy@d@lx#1\endcsname##1{%
    \csname phy@#1@\string##1\endcsname##1}}
%    \end{macrocode}
% \begin{function}{\phy@d@lxmb, \phy@d@lxab}
%   \begin{syntax}
%     \cs{phy@d@lxmb} \meta{biggg or *} \meta{left delimiter} \meta{subformula} \meta{right delimiter}
%     \cs{phy@d@lxab} \phantom{\meta{biggg or *}} \meta{left delimiter} \meta{subformula} \meta{right delimiter}
%   \end{syntax}
% \end{function}
%    \begin{macrocode}
\phy@d@l@genxm{mb}
\phy@d@l@genxa{ab}
%    \end{macrocode}
% \begin{function}{\ab}
%   The users' command \cs{ab}.
% \end{function}
%    \begin{macrocode}
\DeclareRobustCommand\ab{\phy@d@lx{mb}{ab}}
%    \end{macrocode}
% \subsection{\cs{pab}-like commands}
% This is so simple. No need to comment a lot.
% 
% \begin{function}{\phy@d@l@geny}
%   \begin{syntax}
%     \cs{phy@d@l@geny} \meta{command} \meta{left delimiter} \meta{right delimiter}
%   \end{syntax}
%   This command used to define commands like \cs{pab}.
% \end{function}
%    \begin{macrocode}
\def\phy@d@l@geny#1#2#3{%
  \DeclareDocumentCommand#1{so+m}{% ##1: star; ##2: bigg (csname); ##3: subformula.
    \IfBooleanTF{##1}%
      {#2##3#3}%
      {\IfValueTF{##2}%
        {\csname##2l\endcsname#2##3\csname##2r\endcsname#3}%
        {\phy@abopen#2##3\phy@abclose#3}%
      }%
  }%
}
\phy@d@l@geny\pab()
\phy@d@l@geny\bab[]
\phy@d@l@geny\Bab\lbrace\rbrace
\phy@d@l@geny\vab\vert\vert
\phy@d@l@geny\aab\langle\rangle
\phy@d@l@geny\Vab\Vert\Vert
%</ab>
%    \end{macrocode}
%
%
%
% \section{The \modu{ab.legacy} module}
%    \begin{macrocode}
%<*ab.legacy>
\ProvidesFile{phy-ab.legacy.sty}
  [2023/10/24 `ab.legacy' module of physics2]
%    \end{macrocode}
% Requires \modu{ab}'s |tight| option.
%    \begin{macrocode}
\phy@requiremodule{ab}
\phy@define@key{ab.legacy}{order}[\mathcal{O}]{\def\phy@ab@ordersym{#1}}
\phy@setkeys{ab.legacy}{order}
\phy@processkeyopt{ab.legacy}
\phy@d@l@geny\abs\vert\vert
\phy@d@l@geny\norm\Vert\Vert
\DeclareDocumentCommand\order{som}{%
  \phy@ab@ordersym
  \IfBooleanTF{#1}
    {(#3)}
    {\IfValueTF{#2}
      {\csname#2l\endcsname(#3\csname#2r\endcsname)}
      {\phy@abopen(#3\phy@abclose)}%
    }%
}
\phy@d@l@geny\eval.\vert
\phy@d@l@geny\peval(\vert
\phy@d@l@geny\beval[\vert
%</ab.legacy>
%    \end{macrocode}
\endinput
