% \DeclareSourceFile{base}
%
% This file describes the \pkg{physics2} package and its required patch file
% (module \modu{patch.amsmath}).
%
% \section{The \pkg{physics2} package}
%
%    \begin{macrocode}
%<@@=phy>
%    \end{macrocode}
%
% \subsection{How to write a module}
%
% The \pkg{physics2} package itself provides a standard interface for loading
% modules. A module is a pure text file that contains \TeX\ code. One module
% provides one bundle of functions. The file name of a module is like
% \texttt{phy-\meta{module}.sty}. It's recommended that functions in a module
% are aimed at a specific topic and share similar usage. Modules can handle
% key-value options.
%
% The beginning of a module's content should be like
% \begin{verbatim}
% \ProvidesFile{phy-MODULE_NAME.sty}[YYYY/MM/DD DESCRIPTION]
% \end{verbatim}
% where \texttt{phy-MODULE_NAME.sty} is the same as the file name of the module.
% After this line, other packages or \pkg{physics2} modules can be required,
% and options of this module can be processed.
%
% For example, the beginning of a module might look like
% \begin{verbatim}
% \ProvidesFile{phy-something.sty}
%   [2025/01/01 This module does something]
% \RequirePackage{...}
% \phy@requiremodule{...}
% \DeclareKeys{bool-option.if=phy@ModBoolA}
% \SetKeys{bool-option=true}
% \ProcessKeyOptions
% \ifphy@ModBoolA
%   ...
% \end{verbatim}
% It should be explained that a module's options can be declared with
% \cs{DeclareKeys}, which is added to \LaTeX\ kernel since release 2022-06-01.
% The default values of keys can be set with \cs{SetKeys}, and users' settings
% can be processed with \cs{ProcessKeyOptions}. If you are not familiar with
% such procedures, you can run \texttt{texdoc clsguide} to get more information.
%
% If you are writing a module in \LaTeXe\ ``traditional'' syntax, avoid using
% two consecutive ``\texttt{@}'' symbols in your internal commands. Two
% consecutive ``\texttt{@}'' symbols may cause confusion to module authors
% using \LaTeX3 syntax and DocStrip utility.
%
% \subsection{API}\label{subsec:api.physics2}
%
% \begin{variable}{\phy@tempcnta,\phy@tempdima,\phy@tempdimb,\phy@tempskipa,
%   \phy@tempmuskipa,\phy@tempboxa,\ifphy@tempswa,\phy@temptoksa}
% Common temporary variables that can be shared by any module.
% \end{variable}
%
% \begin{variable}{\l_@@_tmpa_int,\l_@@_tmpb_int,\l_@@_tmpa_tl,\l_@@_tmpb_tl}
% Common temporary variables (L3 style) that can be shared by any module.
% \end{variable}
%
% \begin{function}{\usephysicsmodule,\phy@requiremodule}
% \begin{syntax}
% \cs{usephysicsmodule} \oarg{key-val options} \marg{module} \oarg{date}
% \cs{phy@requiremodule} \oarg{key-val options} \marg{module} \oarg{date}
% \end{syntax}
% Loads modules. The usage is the same as \cs{RequirePackage}.
% \cs{usephysicsmodule} is a user command, and \cs{phy@requiremodule} is a
% developer command.
% \relatedint{phy@modulewithdate}
% \end{function}
%
% \begin{variable}{\phy@ModBoolA,\phy@ModBoolB}
% Booleans used when handling key-value options of a module. They are set false
% when a module is loaded.
% \end{variable}
%
% \begin{function}{\phy@ModuleError,\phy@ModuleWarning,
%   \@@_module_error:nnn,\@@_module_warning:nn}
% \begin{syntax}
% \cs{phy@ModuleError} \marg{module} \marg{error message} \marg{help message}
% \cs{phy@ModuleWarning} \marg{module} \marg{warning message}
% \cs{@@_module_error:nnn} \marg{module} \marg{error message} \marg{help message}
% \cs{@@_module_warning:nn} \marg{module} \marg{warning message}
% \end{syntax}
% Error and warning handlers for modules of this package.
% \end{function}
%
% \begin{function}{\delopen,\delclose}
% \begin{syntax}
% \cs{delopen} \meta{left delimiter} \meta{math mode material} \cs{delclose} \meta{right delimiter}
% \end{syntax}
% Re-packaged \cs{left} and \cs{right}. A \cs{left} and \cs{right} pair will
% be treated as an inner formula, around which 3\,mu skips are inserted. If
% you don't want the skips, use \cs{delopen} and \cs{delclose} instead.
% \end{function}
%
% \begin{function}{\phy@mathvphantom,\@@_math_vphantom:n}
% \begin{syntax}
% \cs{phy@mathvphantom} \marg{math mode material}
% \cs{@@_math_vphantom:n} \marg{math mode material}
% \end{syntax}
% A math mode version of \cs{vphantom}.
% \relatedint{phy@mathvph@ntom}
% \end{function}
%
% \subsection{Internal commands}
%
% \begin{function}{\phy@modulewithdate}
% \begin{syntax}
% \cs{phy@modulewithdate} \oarg{keyval options} \marg{module} \oarg{date}
% \end{syntax}
% Loads a module with keyval options and a specific date.
% \relatedint{phy@ResetModBools}
% \begin{NOTE}
% Hint:
% \cs{@onefilewithoptions}\marg{filename}\oarg{options}\oarg{date}\marg{extension}
% \end{NOTE}
% \end{function}
%
% \begin{function}{\phy@ResetModBools}
% \begin{syntax}
% \cs{phy@ResetModBools}
% \end{syntax}
% Sets all booleans for handling modules' key-value options (like \cs{phy@ModBoolA})
% false. It's called every time a separate module is loaded.
% \end{function}
%
% \begin{function}{\phy@m@thvphantom}
% \begin{syntax}
% \cs{phy@m@thvphantom} \meta{math style command} \marg{math mode material}
% \end{syntax}
% Generate a zero-width box with the height and depth of the math material in a
% specific math style. \meta{math style command} is \cs[no-index]{displaystyle}, etc.
% \end{function}
%
% \begin{function}{\@@_reset_math_class:nnn}
% \begin{syntax}
% \cs{@@_reset_math_class:nn} \marg{charcode} \marg{previous math class} \marg{new math class}
% \end{syntax}
% Incomplete.
% \end{function}
%
% \begin{function}{\@@_patch_unimath:}
% \begin{syntax}
% \cs{@@_patch_unimath:}
% \end{syntax}
% Patch for \pkg{unicode-math}. The current functions are
% \begin{itemize}
% \item Making \cs{vert} different from ``\texttt{\char`\|}'' in math mode.
% \end{itemize}
% See \S\ref{subsec:code.base} for details.
% \end{function}
%
% \begin{function}{\@@_patch_no_unimath_common:,\@@_patch_mathtools:}
% \begin{syntax}
% \cs{@@_patch_no_unimath_common:}
% \end{syntax}
% See \S\ref{subsec:code.base} for details.
% \end{function}
%
% \begin{function}{\@@_patch_amsmath:}
% \end{function}
%
% \subsection{Code implementation}\label{subsec:code.base}
%
%    \begin{macrocode}
%<*package>
\NeedsTeXFormat{LaTeX2e}[2024/11/01]
\ProvidesPackage{physics2}
  [2025/03/01 v2.0.0 Tools for typesetting math formulae.]
%    \end{macrocode}
%
% \begin{macro}{\phy@ModuleError}
%    \begin{macrocode}
\gdef\phy@ModuleError#1#2#3{%
  \GenericError{%
    (phy-#1)\@spaces\@spaces\@spaces\@spaces
  }{%
    Module #1 (of physics2) Error: #2%
  }{%
     See the physics2 package documentation (texdoc physics2) for explanation.%
  }{#3}%
}
%    \end{macrocode}
% \end{macro}
%
% \begin{macro}{\phy@ModuleWarning}
%    \begin{macrocode}
\def\phy@ModuleWarning#1#2{%
  \GenericWarning{%
    (phy-#1)\@spaces\@spaces\@spaces\@spaces
  }{%
    Module #1 (of physics2) Warning: #2%
  }%
}
%    \end{macrocode}
% \end{macro}
%
% \begin{macro}[int]{\phy@tempcnta,\phy@tempdima,\phy@tempdimb,
%                    \phy@tempskipa,\phy@tempmuskipa,\phy@tempboxa,
%                    \ifphy@tempswa,\phy@temptoksa}
%    \begin{macrocode}
\newcount  \phy@tempcnta
\newdimen  \phy@tempdima
\newdimen  \phy@tempdimb
\newskip   \phy@tempskipa
\newmuskip \phy@tempmuskipa
\newbox    \phy@tempboxa
\newif     \ifphy@tempswa
\newtoks   \phy@temptoksa
%    \end{macrocode}
% \end{macro}
%
% \begin{macro}[int]{\phy@ModBoolA,\phy@ModBoolB}
%    \begin{macrocode}
\newif     \ifphy@ModBoolA
\newif     \ifphy@ModBoolB
%    \end{macrocode}
% \end{macro}
%
% \begin{macro}{\phy@ResetModBools}
% See \cs{phy@modulewithdate}.
%    \begin{macrocode}
\protected\def\phy@ResetModBools{%
  \global\phy@ModBoolAfalse
  \global\phy@ModBoolBfalse
}
%    \end{macrocode}
% \end{macro}
%
% \begin{macro}{\usephysicsmodule, \phy@requiremodule}
%    \begin{macrocode}
\def\usephysicsmodule{\@ifnextchar[%]
  {\phy@modulewithopt}{\phy@modulewithopt[]}}
\let\phy@requiremodule\usephysicsmodule
\@onlypreamble\usephysicsmodule
\def\phy@modulewithopt[#1]#2{\@ifnextchar[%]
  {\phy@modulewithdate[{#1}]#2}{\phy@modulewithdate[{#1}]#2[]}}
%    \end{macrocode}
% \end{macro}
%
% \begin{macro}{\phy@modulewithdate}
%    \begin{macrocode}
\def\phy@modulewithdate[#1]#2[#3]{%
  \def\reserved@b##1,{%
    \ifx\@nnil##1\relax\else
      \ifx\@nnil##1\@nnil\else
        \phy@ResetModBools
        \noexpand\@onefilewithoptions{phy-##1}[{\unexpanded{#1}}][{#3}]%
        \noexpand\@pkgextension
      \fi
      \expandafter\reserved@b
    \fi}%
    \edef\reserved@a{\zap@space#2 \@empty}%
    \edef\reserved@a{\expandafter\reserved@b\reserved@a,\@nnil,}%
  \reserved@a}
%    \end{macrocode}
% \end{macro}
%
% \begin{macro}{\phy@mathvphantom,\phy@mathvph@ntom}
%    \begin{macrocode}
\def\phy@mathvphantom#1{%
  \setbox\phy@tempboxa=\hbox{}%
  \mathpalette\phy@mathvph@ntom{#1}%
}
\def\phy@mathvph@ntom#1#2{%
  \setbox\@tempboxa\hbox{$#1#2$}%
  \ht\phy@tempboxa=\ht\@tempboxa
  \dp\phy@tempboxa=\dp\@tempboxa
  \box\phy@tempboxa
}
%    \end{macrocode}
% \end{macro}
%
% \begin{macro}{\delopen,\delclose}
%    \begin{macrocode}
\DeclareRobustCommand\delopen{\mathopen{}\mathclose\bgroup\left}
\DeclareRobustCommand\delclose{\aftergroup\egroup\right}
%    \end{macrocode}
% \end{macro}
%
%    \begin{macrocode}
\ExplSyntaxOn
%    \end{macrocode}
%
% \begin{macro}[int]{\l_@@_tmpa_int,\l_@@_tmpb_int,\l_@@_tmpa_tl,\l_@@_tmpb_tl}
%    \begin{macrocode}
\int_new:N \l_@@_tmpa_int
\int_new:N \l_@@_tmpb_int
\tl_new:N  \l_@@_tmpa_tl
\tl_new:N  \l_@@_tmpb_tl
%    \end{macrocode}
% \end{macro}
% \begin{macro}{\@@_module_error:nnn,\@@_module_warning:nn}
%    \begin{macrocode}
\cs_set_eq:NN \@@_module_error:nnn  \phy@ModuleError
\cs_set_eq:NN \@@_module_warning:nn \phy@ModuleWarning
%    \end{macrocode}
% \end{macro}
%
% \begin{macro}{\@@_math_vphantom:n}
%    \begin{macrocode}
\cs_set_eq:NN \@@_math_vphantom:n   \phy@mathvphantom
%    \end{macrocode}
% \end{macro}
%
% \begin{macro}{\DeclareBiggg,\@@_declare_biggg:nn}
%    \begin{macrocode}
\cs_new_nopar:Npn \DeclareBiggg #1#2
  {
    \AddToHook { begindocument/before }
      { \@@_declare_biggg:nn {#1} {#2} }
  }
\cs_new_nopar:Npn \@@_declare_biggg:nn #1#2
  {
    \DeclareRobustCommand#1{\bBigg@{#2}}%
    \edef\phy@tempa{\expandafter\@gobble\string#1}%
    \expandafter\DeclareRobustCommand\csname\phy@tempa l\endcsname{\mathopen#1}%
    \expandafter\DeclareRobustCommand\csname\phy@tempa m\endcsname{\mathrel#1}%
    \expandafter\DeclareRobustCommand\csname\phy@tempa r\endcsname{\mathclose#1}%
    \expandafter\def\csname phy@ab@chk\string#1\endcsname{}%
  }
%    \end{macrocode}
% \end{macro}
%
% \begin{macro}{\@@_declare_biggg:}
%    \begin{macrocode}
\cs_new_nopar:Npn \@@_declare_biggg:
  {
    \expandafter\def\csname phy@ab@chk\string*\endcsname{}
    \expandafter\def\csname phy@ab@chk\string\big\endcsname{}
    \expandafter\def\csname phy@ab@chk\string\Big\endcsname{}
    \expandafter\def\csname phy@ab@chk\string\bigg\endcsname{}
    \expandafter\def\csname phy@ab@chk\string\Bigg\endcsname{}
    \@@_declare_biggg:nn \biggg { 3 }
    \@@_declare_biggg:nn \Biggg { 3.5 }
  }
%    \end{macrocode}
% \end{macro}
%
% \begin{macro}{\@@_reset_math_class:nnn}
%    \begin{macrocode}
\cs_new_nopar:Npn \@@_reset_math_class:nnn #1 #2 #3
  { }
%    \end{macrocode}
% \end{macro}
%
% \begin{macro}{\@@_patch_unimath:}
% \pkg{unicode-math} makes control sequences for math symbols the same as the
% symbols themselves. For example, \cs{vert} and ``\texttt{\char`\|}'' are
% \cs{cs_gset_eq:NN}'ed (or \cs{global}\cs{let}) to be same.
% See \cs{__um_define_math_chars:} in \texttt{um-code-setchar.dtx}.
%
% In \modu{ab.braket} module of this package, we need \cs{vert} and
% ``\texttt{\char`\|}'' to be different. So here we make \cs{vert} defined to
% be a ``\texttt{\char`\|}'' with catcode 12 with \cs{xdef}. Then the meaning
% of ``\texttt{\char`\|}'' is character ``\texttt{\char`\|}'' itself, and the
% meaning of \cs{vert} is a \cs{protected}\cs{long} macro whose expansion is
% ``\texttt{\char`\|}''.
%    \begin{macrocode}
\cs_new_nopar:Npn \@@_patch_unimath:
  {
    \cs_gset_protected_nopar:Npe \vert { \char_generate:nn {"007C} {12} }
  }
%    \end{macrocode}
% \end{macro}
%
% \begin{macro}{\@@_patch_no_unimath_common:}
% In \LaTeXe\ kernel, \cs{vert} and \cs{|} are defined to be same, with
% \cs{DeclareMathDelimiter}. In \modu{ab.braket} module of this package, we
% also need \cs{vert} and \cs{|} to be different. So we use the same method
% as \cs{@@_patch_unimath:}.
%    \begin{macrocode}
\cs_new_nopar:Npn \@@_patch_no_unimath_common:
  {
    \cs_gset_protected_nopar:Npn \| { \Vert }
  }
%    \end{macrocode}
% \end{macro}
%
% \begin{macro}{\@@_patch_amsmath:}
%    \begin{macrocode}
\cs_new_nopar:Npn \@@_patch_amsmath:
  {
    \phy@requiremodule { patch.amsmath }
  }
%    \end{macrocode}
% \end{macro}
%
% \begin{macro}{\@@_patch_mathtools:}
% \cs{lparen} and \cs{rparen} are defined in \pkg{mathtools} or
% \pkg{unicode-math}. If users didn't load any of them, we define them here.
% The reason why we need to define them is that modules like \modu{ab} need
% them.
%    \begin{macrocode}
\cs_new_nopar:Npn \@@_patch_mathtools:
  {
    \protected\def\lparen{(}
    \protected\def\rparen{)}
  }
%    \end{macrocode}
% \end{macro}
%
% Modules of \pkg{physics2} would use the following math symbol commands:
% \begin{center}
% \begin{minipage}{.6\textwidth}
% \def\from#1{\textsuperscript{\textit{\textsf{#1}}}}%
% \hfil
% \begin{tabular}{lll}
% \hline
% & \cs{lparen}\from{m} & \cs{rparen}\from{m} \\
% & \cs{lbrack}         & \cs{rbrack}         \\
% & \cs{lbrace}         & \cs{rbrace}         \\
% & \cs{langle}         & \cs{rangle}         \\
% \cs{vert} & \cs{lvert}\from{a} & \cs{rvert}\from{a} \\
% \cs{Vert} & \cs{lVert}\from{a} & \cs{rVert}\from{a} \\
% \hline
% \end{tabular}%
% \hfil\par
% \from{a}Defined in \pkg{amsmath}   \par
% \from{m}Defined in \pkg{mathtools} \par
% \end{minipage}
% \end{center}
% If users load \pkg{unicode-math}, \pkg{unicode-math} would define these
% commands. If not, note that we cannot use them directly because some of
% them are not defined in \LaTeXe\ kernel. Some of them are defined in
% \pkg{amsmath} and some are defined in \pkg{mathtools}. We need to define
% those undefined in \pkg{physics2}. \pkg{mathtools} loads \pkg{amsmath}
% automatically, so those defined in \pkg{amsmath} needn't be defined again
% if \pkg{mathtools} is loaded. But if \pkg{amsmath} is loaded but \pkg{mathtools}
% is not, we need to define those defined in \pkg{mathtools} but not in
% \pkg{amsmath}. We need to make a patch for this, whose code is to define
% these commands if one or some of them are not defined, that is, if at least
% one of \pkg{unicode-math}, \pkg{mathtools} and \pkg{amsmath}, is not loaded.
% First we need to check if \pkg{unicode-math} is loaded, then \pkg{mathtools},
% and finally \pkg{amsmath}. So we get
% \begin{verbatim}
% \IfPackageLoadedF { unicode-math }
%   {
%     \IfPackageLoadedF { mathtools }
%       {
%         \@@_patch_mathtools:   % -> to define \lparen, \rparen
%         \IfPackageLoadedF { amsmath }
%           {
%             \@@_patch_amsmath: % -> to define \lvert, \rvert, \lVert, \rVert
%           }
%       }
%   }
% \end{verbatim}
% Those packages could be loaded after \pkg{physics2}, so it's better to delay
% the code to the end of preamble.
%
% \pkg{unicode-math} can be loaded before or after \pkg{physics2}, so if we
% want to check if \pkg{unicode-math} is loaded, it's better to delay the code
% to the end of preamble.
%
% What's more, \pkg{unicode-math} delays the definition
% of math symbols (\cs{__um_define_math_chars:}) after the \env{document}
% environment begins. In order to make the patch work, we need to delay the
% patch after \cs{__um_define_math_chars:}. That's why we use \cs{AtBeginDocument}
% in the true branch of \cs{IfPackageLoadedTF} in the following code.
% 
% If \pkg{unicode-math} is loaded, execute \cs{@@_patch_unimath:}.
% \cs{@@_patch_unimath:} contains code to redefine math symbol commands, 
% and \pkg{unicode-math} defines math symbols with \cs{__um_define_math_chars:}
% after \env{document} begins. So we need to execute \cs{@@_patch_unimath:}
% after \cs{__um_define_math_chars:}. That's why we use \cs{AtBeginDocument}
% in the following code.
%    \begin{macrocode}
\AddToHook { begindocument/before }
  {
    \IfPackageLoadedTF { unicode-math }
      {
        \AtBeginDocument { \@@_patch_unimath: }
      }
      {
        \@@_patch_no_unimath_common:
        \IfPackageLoadedF { mathtools }
          {
            \@@_patch_mathtools:
            \IfPackageLoadedF { amsmath }
              {
                \@@_patch_amsmath:
              }
          }
      }
    \@@_declare_biggg:
  }
%    \end{macrocode}
%
%    \begin{macrocode}
\ExplSyntaxOff
%<@@=>
%</package>
%    \end{macrocode}
%
% \subsection{Patch for \pkg{amsmath}}\label{subsec:patch.amsmath}
%    \begin{macrocode}
%<*patch.amsmath>
\ProvidesFile{phy-patch.amsmath.sty}[2025/03/01 physics2 patch for amsmath]
%    \end{macrocode}
%
% \begin{macro}{\lvert, \rvert, \lVert, \rVert}
% The following code comes from \S7.2 (Vertical bar symbols)
% of \texttt{amsmath.pdf}.
% 
% \cs{@xp} is replaced by \cs{expandafter} and
% \cs{ifx}\cs{@@undefined}\cs[no-index]{\meta{cs}} is replaced by
% \cs{ifdefined}\cs[no-index]{\meta{cs}}\cs{else}.
%    \begin{macrocode}
\def\@tempa#1#2\@nil{%
  \ifx\delimiter#1\@tempcnta#2\relax\else\@tempcnta\z@\fi
}
\expandafter\@tempa\vert\@empty\@nil
\ifnum\@tempcnta>\z@
  \advance\@tempcnta "4000000
  \protected\xdef\lvert{\delimiter\number\@tempcnta\space }
  \advance\@tempcnta "1000000
  \protected\xdef\rvert{\delimiter\number\@tempcnta\space }
\else
  \ifdefined\lvert\else
    \DeclareMathDelimiter\lvert
      {\mathopen}{symbols}{"6A}{largesymbols}{"0C}
    \DeclareMathDelimiter\rvert
      {\mathclose}{symbols}{"6A}{largesymbols}{"0C}
  \fi
\fi
\expandafter\@tempa\Vert\@empty\@nil
\ifnum\@tempcnta>\z@
  \advance\@tempcnta "4000000
  \protected\xdef\lVert{\delimiter\number\@tempcnta\space }
  \advance\@tempcnta "1000000
  \protected\xdef\rVert{\delimiter\number\@tempcnta\space }
\else
  \ifdefined\lVert\else
    \DeclareMathDelimiter\lVert
      {\mathopen}{symbols}{"6B}{largesymbols}{"0D}
    \DeclareMathDelimiter\rVert
      {\mathclose}{symbols}{"6B}{largesymbols}{"0D}
  \fi
\fi
%    \end{macrocode}
%\end{macro}
% \begin{macro}{\resetMathstrut@,\@mathmeasure,\bBigg@}
%    \begin{macrocode}
\newdimen\phy@patch@bigsize
\newbox\phy@patch@mathstrutbox
\setbox\phy@patch@mathstrutbox=\hbox{}
\ifx\Umathcharnumdef\@undefined
  \def\resetMathstrut@{%
    \begingroup
    \setbox\z@\hbox{%
      \mathchardef\@tempa\mathcode`\(\relax
      \def\@tempb##1"##2##3{\the\textfont"##3\char"}%
      \expandafter\@tempb\meaning\@tempa \relax
    }%
    \edef\@tempa{%
      \ht\phy@patch@mathstrutbox\the\ht\z@\relax
      \dp\phy@patch@mathstrutbox\the\dp\z@\relax}%
    \expandafter\endgroup\@tempa
  }
\else
  \def\resetMathstrut@{%
    \begingroup
    \Umathcharnumdef\@tempa\Umathcodenum`\(\relax
    \def\@tempb##1"##2"##3"##4\relax{%
      \endgroup
      \ht\phy@patch@mathstrutbox=\fontcharht\textfont"##3 "##4\relax
      \dp\phy@patch@mathstrutbox=\fontchardp\textfont"##3 "##4\relax}%
    \expandafter\@tempb\meaning\@tempa \relax
  }
\fi
\addto@hook\every@math@size{\resetMathstrut@}
\addto@hook\every@math@size{%
  \global\phy@patch@bigsize 1.2\ht\phy@patch@mathstrutbox
  \global\advance\phy@patch@bigsize 1.2\dp\phy@patch@mathstrutbox }
\def\@mathmeasure#1#2#3{\setbox#1\hbox{%
  \m@th$#2#3$}}
\def\bBigg@#1#2{\leavevmode@ifvmode
  {\@mathmeasure\z@{\nulldelimiterspace\z@}%
     {\left#2\vcenter to#1\phy@patch@bigsize{}\right.}%
  \box\z@}}
\DeclareRobustCommand\big{\bBigg@{1}}
\DeclareRobustCommand\Big{\bBigg@{1.5}}
\DeclareRobustCommand\bigg{\bBigg@{2}}
\DeclareRobustCommand\Bigg{\bBigg@{2.5}}
%    \end{macrocode}
% \end{macro}
%    \begin{macrocode}
%</patch.amsmath>
%    \end{macrocode}
\endinput
