%% ******************************************************
%% * This work may be distributed and/or modified under *
%% * the conditions of the LaTeX Project Public License *
%% *     http://www.latex-project.org/lppl.txt          *
%% * either version 1.3c of this license or any later   *
%% * version.                                           *
%% ******************************************************
\documentclass[11pt,letterpaper]{article}
\usepackage{amsmath}
\usepackage{unicode-math}
\setmainfont{Libertinus Serif}
\setsansfont{Libertinus Sans}
\setmonofont{Noto Sans Mono}[Scale=MatchLowercase]
\setmathfont{Libertinus Math}
\usepackage{physics2}
\usephysicsmodule{ab,ab.braket}
\usephysicsmodule{diagmat}
\usephysicsmodule{doubleprod}
\usephysicsmodule[showleft=3,showtop=3]{xmat}
\usephysicsmodule{ab.legacy,nabla.legacy,op.legacy,qtext.legacy,bm-um.legacy}
%% This is part of physics2 documentation.
%% ******************************************************
%% * This work may be distributed and/or modified under *
%% * the conditions of the LaTeX Project Public License *
%% *     http://www.latex-project.org/lppl.txt          *
%% * either version 1.3c of this license or any later   *
%% * version.                                           *
%% ******************************************************
\makeatletter
\usepackage[nohyperref]{doc}
\renewcommand*\l@subsection{\@dottedtocline{2}{1.5em}{2.1em}}
\def\contentsonly{\@starttoc{toc}}
\usepackage{manfnt}
\protected\def\pardanger{\ifnum\lastnodetype=1%
  \leavevmode\hbox to \z@{\hss\dbend\hskip10pt\hskip\parindent}\else
  \ifnum\lastnodetype=11%
  \leavevmode\hbox to \z@{\hss\dbend\hskip10pt}%
  \fi\fi\ignorespaces}
\usepackage{hyperref,color,xr}
\definecolor{pkgcolor}{RGB}{0,120,80}
\definecolor{moducolor}{RGB}{0,0,150}
\hypersetup{colorlinks,urlcolor=magenta,linkcolor=black,filecolor=blue,
  pdfstartview=FitH,pdfview=FitH,pdfcreator=XeTeX output}
\def\s@pkg#1{\texorpdfstring{\textcolor{pkgcolor}{\textsf{#1}}}{“#1”}}
\def\@pkg#1{\texorpdfstring{\href{https://www.ctan.org/pkg/#1}%
  {\textcolor{pkgcolor}{\textsf{#1}}}}{“#1”}}
\DeclareRobustCommand\pkg{\@ifstar\@pkg\s@pkg}
\def\modu#1{\texorpdfstring{\textcolor{moducolor}{\textsf{#1}}}{“#1”}}
\def\opt#1{\texorpdfstring{\texttt{#1}}{“#1”}}
\protected\def\marg#1{\texttt\{\meta{#1}\texttt\}}
\protected\def\oarg#1{\texttt[\meta{#1}\texttt]}
\parindent2.4em
\linespread{1.06}
\usepackage{multicol,fancyvrb,verbatim,array,hologo}
\fvset{xleftmargin=\parindent}
\newenvironment{displayed}{\trivlist\item[]\leftskip\parindent\relax}{\endtrivlist}
\usepackage[normalem]{ulem}
\newcounter{example}[subsection]
\renewcommand\theexample{\thesubsection.\arabic{example}}
\newwrite\example@out
\def\example@name{\jobname.example.aux}
\long\def\example@start{\begingroup\@bsphack
  \immediate\openout\example@out=\example@name
  \let\do\@makeother\dospecials\catcode`\^^M\active
  \def\verbatim@processline{\immediate\write\example@out{\the\verbatim@line}}%
  \verbatim@start}
\long\def\example@end{\immediate\closeout\example@out\@esphack\endgroup%
  \trivlist\item[]\relax
    \leavevmode\hbox to \z@{%
      \hbox to \z@{\hss{\footnotesize[\theexample]}\hskip10pt}%
      \hskip\parindent
      \begin{minipage}[c]{0.55\textwidth}%
        \small\verbatiminput{\example@name}%
      \end{minipage}%
      \fbox{%
        \begin{minipage}[c]{0.5\textwidth}%
          \normalsize\input{\example@name}%
        \end{minipage}%
      }%
    \hss}%
  \endtrivlist}
\newenvironment{example*}[1]%
  {\refstepcounter{example}\label{#1}\example@start}{\example@end}
\newenvironment{example}
  {\stepcounter{example}\example@start}{\example@end}
\author{Zhang Tingxuan}
\date{2023/04/02\quad Version 0.2.1\thanks{%
  \url{https://www.github.com/AlphaZTX/physics2}}}
\makeatother
\endinput

\externaldocument[lega-]{physics2-legacy}
\title{The \pkg{physics2} package}
\begin{document}
\maketitle

\begin{abstract}
This is the document for \pkg{physics2} package, which defines commands for
typesetting math formulae faster and simpler. \pkg{physics2} is a modularized
package, each module provides its own function. 

This document describes the \pkg{physics2} package in more detail.
But if you are a user of the legacy \pkg{physics} package, you can click
\hyperref{./physics2-legacy.pdf}{}{}{here} to see the documention for
\pkg{physics} users before you start. If you never used \pkg{physics} package
before, just read \emph{this} documentation.
\end{abstract}

\tableofcontents

\section{Introduction}
\subsection{The purpose of this package}
This package aims to provide a bundle of commands for typesetting math faster
in different modules. The commands provided by \pkg{physics2} and its different
modules are designed to be short and easy to memorize.

\subsection{Packages required}
The \pkg{physics2} package itself only requires the \pkg*{keyval} package, which
is part of the \pkg*{latex-graphics} bundle. Almost every \LaTeX\ distribution
will include this bundle.

Different modules of \pkg{physics2} might require different packages. It will
be explained in the following sections corresponding to each module that which
module requires which package.

The \pkg{physics2} package requires \LaTeXe\ kernel released after 2020/10.
Please make sure that your \LaTeX\ distribution is not too old.

\subsection{Loading the \pkg{physics2} package}
Just like loading any package, write
\begin{Verbatim}
\usepackage{physics2}
\end{Verbatim}
in the preamble to load the \pkg{physics2} package. In this version,
\pkg{physics2} doesn't provide a package option.

However, \pkg{physics2} itself only provides very few functions. Actually, it
just provides a method to load modules. You need to load different modules of
\pkg{physics2} to have different kinds of functions applied to your document.

\subsection{Loading a module of \pkg{physics2}}
You can load a module of \pkg{physics2} only \emph{after} you write
\verb|\usepackage{physics2}| in the preamble. Load a \pkg{physics2} module
like this:
\begin{center}
\cs{usephysicsmodule}\marg{module}
\end{center}
The usage of \cs{usephysicsmodule} is similar to \cs{usepackage}, so you can
load more than one modules in one line. For example,
\begin{Verbatim}
\usephysicsmodule{ab,ab.braket}
\end{Verbatim}
This line loads the \modu{ab} and \modu{ab.braket} modules.

You can also load \emph{one} module with options. The options of a
\pkg{physics2} module can be a comma-separated key-value list. For example,
\begin{Verbatim}
\usephysicsmodule[tightbraces=true]{ab}
\usephysicsmodule{ab.braket,doubleprod}
\end{Verbatim}
These two lines load the \modu{ab} module with option
\opt{tightbraces\:=\:true} and load \modu{ab.braket} and \modu{doubleprod}
modules.

\pardanger
The \modu{common} module will be loaded automatically when you load the
\pkg{physics2} package and \emph{only} the \modu{common} module will be
loaded automatically. Any other module should be loaded manually by writing
\cs{usephysicsmodule}\marg{module} after you loaded \pkg{physics2} in the
preamble.

The following sections introduce all the user-level modules of \pkg{physics2}.

\section{Modules of \pkg{physics2}}
\subsection{The automatically loaded \modu{common} module}
The \modu{common} module provides the following commands:

\cs{delopen} and \cs{delclose}, followed by a math delimiter. They can be
regarded as abbreviations of ``open delimiter'' and ``close delimiter''. If
you had heard of the \pkg*{mleftright} package. You can think \cs{delopen}
and \cs{delclose} a simpler version of \cs{mleft} and \cs{mright}. For example,
\begin{example}
\[    0 \left(\frac12\right) 3    \]
\[ 0 \delopen(\frac12\delclose) 3 \]
\end{example}

\cs{biggg} and \cs{Biggg}, followed by a math delimiter. They are even bigger
than \cs{Bigg}. \cs{biggg} and \cs{Biggg} may be useful when you need to write
something really tall in math mode, but most OpenType math font do not support
\cs{langle} (or U+27E8) and \cs{rangle} (or U+27E9) in this large size. Take
an example,
\begin{example}
\[\Biggg(\biggg(\Bigg(\bigg(\Big(\big((
)\big)\Big)\bigg)\Bigg)\biggg)\Biggg)\]
\end{example}
\cs{bigggl}, \cs{bigggm}, \cs{bigggr}, \cs{Bigggl}, \cs{Bigggm} and \cs{Bigggr}
are also supported.

\subsection{The \modu{ab} module --- automatic braces}
This module provides the command \cs{ab}. The \cs{ab} command, as a shorthand
of ``automatic braces'', would specify the size of the following pair of
delimiters' size. The delimiters after \cs{ab} should not be out of the range
described by the following chart:
\begin{center}
\begin{tabular}{c@{\hskip2em}l@{\hskip2em}c}
\opt{(},\quad\opt{)} && \\
\opt{[},\quad\opt{]} && \\
\cs{\{},\quad\cs{\}} &or& \cs{lbrace},\quad\cs{rbrace} \\
\opt{<},\quad\opt{>} &or& \cs{langle},\quad\cs{rangle} \\
\opt{|},\quad\opt{|} &or&   \cs{vert},\quad\cs{vert} \\
 \cs{|},\quad\cs{|}  &or&   \cs{Vert},\quad\cs{Vert}
\end{tabular}
\end{center}
For example, it's illegal to write an ``\verb|\ab(|'' without a ``\verb|)|'';
it's also illegal to write \verb|\ab=foo=|. Take some correct examples:
\begin{example}
\[ \ab ( \frac12 )  \quad
   \ab [ \frac12 ]  \quad
   \ab\{ \frac12 \}    \]
\end{example}
You can also write a command from \cs{big} to \cs{Biggg} between \cs{ab}
and the first delimiter, which means to specify the size of delimiters manually.
Also, you can write a star (\opt{*}) between \cs{ab} and the first delimiter,
to prevent \cs{ab} from setting the size of delimiters. For example,
\begin{example}
\[ \ab      <\frac12> \quad
   \ab\biggg|\frac12| \quad
   \ab*    \|\frac12\|   \]
\end{example}
\pardanger
Always remember, do not put an \cs{ab} separately at the end of math mode like
\verb|$\ab$|, because \cs{ab} will try to absorb the following math shift
character (\verb|$|) as its argument.

The \modu{ab} module also provides \texttt{\textbackslash}$X$\texttt{ab}
commands, where $X$ can be \opt{p}, \opt{b}, \opt{B}, \opt{a}, \opt{v} and
\opt{V}. These commands takes a normal argument but not an argument delimited
with paired delimiters. For example,
\begin{example}
\def\0{\frac12}
\[ \pab{\0} \bab{\0} \Bab{\0} \]
\[ \aab{\0} \vab{\0} \Vab{\0} \]
\end{example}
After \texttt{\textbackslash}$X$\texttt{ab} can be a ``biggg'' command or a
star. For example,
\begin{example}
\def\0{\frac12}
\[ \pab\Big{\0} \quad \bab*{\0} \]
\end{example}

\paragraph{The options of \modu{ab} module}
\opt{tightbraces}, a bool type key, whose default is \opt{true}, influences
whether thin skips are reserved around the paired delimiters. It only works
with the automatically sized delimiters.

\subsection{The \modu{ab.braket} module --- Dirac bra-ket notation}
This module provides four commands --- \cs{bra}, \cs{ket}, \cs{braket} and
\cs{ketbra}. After these commands can be a star (\opt{*}) or a ``biggg''
command. These commands share similar syntaxes like \cs{ab}'s syntax. But,
\emph{the bra-ket commands from \modu{ab.braket} module are completely different
from \cs{ab}}. Their internal structures are different.

The argument of \cs{bra} should be delimited with \opt{<} and \opt{|}, that is,
\begin{center}
\cs{bra} \opt{<} \meta{subformula} \opt{|}
\end{center}
For example,
\begin{example}
\[ \bra < \frac \phi 2 | \]
\[ \bra*< \frac \phi 2 | \]
\[ \bra\Big< \phi  |     \]
\end{example}
The argument of \cs{ket} should be delimited with \opt{|} and \opt{>}, that is,
\begin{center}
\cs{ket} \opt{|} \meta{subformula} \opt{>}
\end{center}
For example,
\begin{example}
\[ \ket | \frac \psi 2 > \]
\[ \ket*| \frac \psi 2 > \]
\[ \ket\Big| \psi >      \]
\end{example}
\pardanger
If you want to write ``$>$'' and ``$<$'' for relations in the argument of
\cs{bra} and \cs{ket}, you can write \verb|\mathrel{>}| and \verb|\mathrel{<}|
(although there is almost no such need).

The argument of \cs{braket} should be delimited with \opt{<} and \opt{>},
that is,
\begin{center}
\cs{braket} \opt{<} \meta{subformula} \opt{>}
\end{center}
In the \meta{subformula} argument, every ``\opt{|}'' will be regarded as an
extensible vertical bar. For example,
\begin{example}
\[ \braket< \phi >            \]
\[ \braket< \phi | \psi >     \]
\[ \braket< \phi | A | \psi > \]
\end{example}
\begin{example}
\def\0{\frac\phi2}
\[ \braket     < \0 | \psi >  \]
\[ \braket*    < \0 | \psi >  \]
\[ \braket\Bigg< \0 | \psi >  \]
\end{example}
The argument of \cs{ketbra} should be delimited with \opt{|} and \opt{|}.
In the argument, \opt{>} and \opt{<} will be regarded as extensible $\rangle$
and $\langle$. that is,
\begin{center}
\cs{ketbra} \opt{|} \meta{subformula$_1$} \opt{>} \meta{optional}
  \opt{<} \meta{subformula$_2$} \opt{|}
\end{center}
For example,
\begin{example}
\def\0{\frac\phi2}
\[ \ketbra     | \0 >< \psi | \]
\[ \ketbra*    | \0 >< \psi | \]
\[ \ketbra\Bigg| \0 >< \psi | \]
\end{example}
\begin{example}
\def\0{\frac\phi2}
\[ \ketbra| \0 >_x^y < \psi | \]
\end{example}
\pardanger
If you want to write ``$>$'' and ``$<$'' for relations in the argument of
\cs{braket} and \cs{ketbra}, you can write \cs{>} and \cs{<} (although there
is almost no such need). It is quite different from \verb|\mathrel{>}| or
\verb|\mathrel{<}| because in these commands' argument, \opt{>} and \opt{<}
will be redefined.

Next, the \modu{braket} module will be introduced. Please notice that
\modu{braket} is conflict with \modu{ab.braket}, they cannot be used together.

\subsection{The \modu{braket} module --- Dirac bra-ket notation}
\begingroup
\makeatletter\def\phy@requiremodule#1{}%
\def\PackageWarning#1#2{}%
\input phy-braket.sty\makeatother
Please notice that this module is conflict with the \modu{ab.braket} module.
Don't use them together.

This module contains four commands --- \cs{bra}, \cs{ket}, \cs{braket} and
\cs{ketbra}. After these commands can be a star (\opt{*}) or a
square-bracket-delimited size option, the size options can
take the following values:
\begin{center}
\opt{big},\quad\opt{Big},\quad\opt{bigg},\quad\opt{Bigg},\quad
\opt{biggg}\quad or\quad\opt{Biggg}.
\end{center}
Star stands for ``do not size the bra-ket automatically''.

The argument(s) of these four commands are braced with \verb|{| and \verb|}|.
\cs{bra} and \cs{ket} take one mandatory argument. For example,
\begin{example}
\def\0{\frac\phi2}
\[ \bra {\0} \quad \bra* {\0}
             \quad \bra[Big] {\0} \]
\[ \ket {\0} \quad \ket* {\0}
             \quad \ket[Big] {\0} \]
\end{example}
The \cs{braket} command, in default, can take two arguments.
\begin{example}
\def\0{\frac\phi2}
\[ \braket {\0} {\psi}   \quad
   \braket*{\0} {\psi}   \quad
   \braket[big] {\0} {\psi} \]
\end{example}
If you want \cs{braket} to take one or three arguments, you can write the
number of arguments in the sqare bracket. If you need to specify the size
of bra-ket simultaneously, you need to separate the number and the size with
a comma:
\begin{example}
\def\0{\frac\phi2}
\[ \braket [1] {\0} \quad
   \braket*[1] {\0} \]
\[ \braket [3] {\0}{A}{\psi}    \]
\[ \braket[3,big] {\0}{A}{\psi}
     \quad
   \braket[Big,3] {\0}{A}{\psi} \]
\end{example}
The \cs{ketbra} command takes two mandatory arguments. It can also take an
optional argument between the two mandatory arguments. The optional argument
will be placed between the $\rangle$ and $\langle$:
\begin{example}
\def\0{\frac\phi2}
\[ \ketbra  {\0} {\psi}    \quad
   \ketbra* {\0} {\psi}       \]
\[ \ketbra [Bigg] {\0} {\psi} \]
\[ \ketbra {\0} [_x^y] {\psi} \]
\end{example}
\endgroup

\subsection{The \modu{diagmat} module --- simple diagonal matrices}
This module provides \cs{diagmat} command:
\begin{center}
\cs{diagmat}\opt{[empty\:=\:}\meta{empty entry}\opt{]}\marg{diag}
\end{center}
where \meta{diag} is the diagonal of the diagonal matrix. The entries should
be separated by commas. The \opt{empty} option is optional, with default value
\opt{0}. For example,
\begin{example}
\[
  \diagmat { 1, 2, 3 }
\]
\end{example}
\cs{pdiagmat}, \cs{bdiagmat}, \cs{Bdiagmat}, \cs{vdiagmat} and \cs{Vdiagmat}
are also available. Prefixes like \opt{p}, \opt{b}, \opt{B} have the same
meaning as the \opt{p}, \opt{b}, \opt{B} in \pkg{amsmath}'s \opt{pmatrix},
\opt{bmatrix} and \opt{Bmatrix}. For example,
\begin{example}
\[
  \pdiagmat [ empty = {} ]
    { a, b, c, d }
\]
\end{example}
This module requires \pkg{amsmath}.

\paragraph{The options of \modu{diagmat} module}
You can set the default value of \cs{diagmat}'s empty entries in the module
option like this:
\begin{Verbatim}
\usephysicsmodule[empty={\cdot}]{diagmat}
\end{Verbatim}

\subsection{The \modu{doubleprod} module --- tensors' double product operator}
Take an example of this module:
\begin{example}
$ A \doublecross B \doubledot C $
\end{example}
\cs{doublecross} and \cs{doubledot} are regarded as binary operators by \TeX.

\paragraph{The options of \modu{doubleprod} module}
You can control the scale of ``$\times$'' and ``$\cdot$'' in \cs{doublecross}
and \cs{doubledot} in module options. For example,
\begin{Verbatim}
\usephysicsmodule[crossscale=0.75,dotscale=1.2]{doubleprod}
\end{Verbatim}
The default values of \opt{crossscale} and \opt{dotscale} are \opt{0.8} and
\opt{1}. You can also control the distances between the two ``$\times$''s and
``$\cdot$''s through the \opt{crossopenup} and \opt{dotopenup} options.
For example,
\begin{Verbatim}
\usephysicsmodule[crossopenup=.05,dotopenup=.25]{doubleprod}
\end{Verbatim}
The default values of \opt{crossopenup} and \opt{dotopenup} are \opt{0.02}
and \opt{0.2}. The value stand for the multiple of current font size.
Moreover, you can change the symbols in \cs{doublecross} and \cs{doubledot}
by setting \opt{crosssymbol} and \opt{dotsymbol} in module options.

\subsection{The \modu{xmat} module --- matrices with formatted entries}
The \modu{xmat} module provides \cs{xmat} command for matrices with formatted
entries:
\begin{center}
\cs{xmat}\oarg{options}\marg{entry}\marg{rows shown}\marg{cols shown}
\end{center}
If \meta{rows shown} and \meta{cols shown} are digits, the value of them
must be less at least 2 than the value of \pkg{amsmath}'s \opt{MaxMatrixCols}
counter. For example,
\begin{example}
\[
  \xmat{a}{2}{3}
\]
\end{example}
\cs{pxmat}, \cs{bxmat}, \cs{Bxmat}, \cs{vxmat} and \cs{Vxmat} are also
available. The meaning of \opt{p} and so on is the same as the \opt{p}
in \opt{pmatrix} of \pkg{amsmath}. For example,
\begin{example}
\[
  \pxmat{M}{3}{3}
\]
\end{example}
If \meta{rows shown} and \meta{cols shown} contain non-digit characters,
extra dots will be added. For example,
\begin{example}
\[
  \bxmat[showleft=3,showtop=2]
    {X}{m}{n}
\]
\end{example}
In this example we used the \opt{showleft} and \opt{showtop} options. The
default value of them is the value of \opt{MaxMatrixCols} minus 2. You can
also set them in the module option like this:
\begin{Verbatim}
\usephysicsmodule[showtop=3,showleft=3]{xmat}
\end{Verbatim}
Then every \cs{xmat} with non-digital \meta{rows shown} and \meta{cols shown}
will have 2 top-most rows and 3 left-most columns shown. This will also
influence digital \meta{rows shown} and \meta{cols shown} \cs{xmat}s when
\meta{rows shown} and \meta{cols shown} are larger than the values
corresponding to \opt{showtop} and \opt{showleft}. For example,
\begin{example}
% \usephysicsmodule
%   [showtop=3,showleft=3]{xmat}
\[ \pxmat{A}{8}{8} \]
\end{example}
However, when \meta{rows shown} and \meta{cols shown} are not \emph{that} large
but still larger than \meta{showtop} and \meta{showleft}, for example,
\meta{rows shown}$\:=4$ and \meta{cols shown}$\:=4$ in last example's settings,
\cs{xmat} will still add the extra dots:
\begin{example}
% \usephysicsmodule
%   [showtop=3,showleft=3]{xmat}
\[ \pxmat{A}{4}{4} \]
\end{example}
In such situations, we need to specify \opt{showtop} and \opt{showleft}
manually. For example,
\begin{example}
% \usephysicsmodule
%   [showtop=3,showleft=3]{xmat}
\[ \pxmat[showtop=4,showleft=4]
     {A}{4}{4}               \]
\end{example}

The \cs{xmat} command provides the \opt{format} option, which allows users
to use a new entry format. For example,
\begin{example}
\[
  \xmat [showleft=2,showtop=2,
    format=\texttt{#1[#2][#3]}]
    {x}{m}{n}
\]
\end{example}
In the value of \opt{format} key, \verb|#1| stands for the common entry,
or the first mandotary (\meta{entry}) argument of \cs{xmat}; \verb|#2|
stands for the row index and \verb|#3| stands for the column index.

This module requires \pkg{amsmath}.

\paragraph{The options of \modu{xmat} module}
Only \opt{showtop} and \opt{showleft} can be used as module options.
\opt{format} should only be used in the optional argument of the
\cs{xmat} command.

\section{The ``legacy'' modules}
The legacy modules have similar names like \meta{module}\opt{.legacy}. Most of
them are designed to provide solutions to maintain documents written with the
legacy \pkg{physics} package. It's not suggest to use them in a new document.

\subsection{The \modu{ab.legacy} module}
This module provides the following commands:
\begin{Verbatim}[fontsize=\small]
\abs   \norm   \eval   \order
\end{Verbatim}
They can take a normal argument. Between these commands and their argument
can be a ``biggg'' command or a star. For example,
\begin{example}
\def\0{\frac12}
\[ \abs{\0}             \quad
   \norm\Big{\0}        \quad
   \order*{\0}             \]
\end{example}
\begin{example}
\def\0{\frac12x}
\[ \eval{\0}_a^b        \quad
   \eval[(]{\0}_a^b     \quad
   \eval[[]\big{\0}_a^b    \]
\end{example}
You can set the ``order'' symbol in this module through
the \opt{order} option like this:
\begin{Verbatim}
\usephysicsmodule[order=O]{ab.legacy}
\end{Verbatim}
For further information of this module, see \S\ref*{lega-subsec:ab-and-legacy}
of \hyperref{./physics2-legacy.pdf}{}{}{physics2-legacy}.

\subsection{The \modu{bm-um.legacy} module}
If you are maintaining a document with plenty of ``\cs{bm}''s or
``\cs{boldsymbol}''s in it but want to use \pkg{unicode-math} package
simultaneously, you could take a look at this module.

The \cs{bm} command from \pkg{bm} package uses \cs{mathversion} to support
its function, but there are few OpenType math fonts who released with a bold
version. The \modu{bm-um.legacy} module provides a \cs{bm} command too, but
this \cs{bm} can only take \emph{one} math character or a series of math
characters sharing the same category code as its argument.
If the argument was a number, \cs{bm} would switch to the bold upright glyph
corresponding to it; else \cs{bm} would switch to the bold italic glyph
corresponding to it if there exists one. For example,
\begin{example}
$\bm{0}\bm{A}\bm{z}
 \bm{\alpha}\bm{\Omega}$
\end{example}

\subsection{The \modu{nabla.legacy} module}
This module provides some commands related to nabla ($\nabla$). Notice that
this module requires the \pkg{fixdif} package with file date 2023/01/31 at
minimum.

This module defines \cs{grad} and \cs{curl} and redefines \cs{div}. For example,
\begin{example}
\[ \grad V      \]
\[ \div (x,y,z) \]
\[ \curl(x,y,z) \]
\end{example}
The ``$\divsymbol$'' symbol was redefined as \cs{divsymbol}.

\subsection{The \modu{op.legacy} module}
This module provides a series of commands for log-like operators. They are
\begin{Verbatim}
\asin   \acos   \atan
\acsc   \asec   \acot
\Tr     \tr     \rank
\erf    \Res    \res
\PV     \pv
\Re     \Im
\end{Verbatim}
where \cs{Re} and \cs{Im} are redefined. The first four lines of commands
yield what they look like in math mode. For example,
\begin{example}
$\asin x$ \quad $\rank A$
\end{example}
\cs{PV} yields ``$\PV$'' as an ordinary symbol and \cs{pv} yields ``$\pv$''.
For example,
\begin{example}
$\PV f(z)$ \quad $\pv f(z)$
\end{example}
\cs{Re} and \cs{Im} are redefined as ``$\Re$'' and ``$\Im$''. $\Resymbol$
and $\Imsymbol$ are redefined as \cs{Resymbol} and \cs{Imsymbol}, in default.

This module \emph{does not} require \pkg{amsmath}.

\paragraph{The options of \modu{op.legacy} module}
\opt{ReIm}, a bool key with dafault true, determines whether to redefine
\cs{Re} and \cs{Im}. If you want to reserved the definition of \cs{Re} and
\cs{Im}, you can write like this:
\begin{Verbatim}
\usephysicsmodule[ReIm=false]{op.legacy}
\end{Verbatim}

\subsection{The \modu{qtext.legacy} module}
This module was written just to offer a method to maintain documents written
with the legacy \pkg{physics} package. See \S\ref*{lega-subsec:qtext} of
\hyperref{./physics2-legacy.pdf}{}{}{physics2-legacy} for more information.

\end{document}