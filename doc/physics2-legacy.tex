%% ******************************************************
%% * This work may be distributed and/or modified under *
%% * the conditions of the LaTeX Project Public License *
%% *     http://www.latex-project.org/lppl.txt          *
%% * either version 1.3c of this license or any later   *
%% * version.                                           *
%% ******************************************************
\documentclass[11pt,letterpaper]{article}
\usepackage{amsmath}
\usepackage{unicode-math}
\setmainfont{Libertinus Serif}
\setsansfont{Libertinus Sans}
\setmonofont{Noto Sans Mono}[Scale=MatchLowercase]
\setmathfont{Libertinus Math}
\usepackage{physics2}
\usephysicsmodule{ab}
\usephysicsmodule{ab.legacy,nabla.legacy}
\usephysicsmodule{op.legacy,qtext.legacy}
\usephysicsmodule{ab.braket}
\usephysicsmodule{diagmat}
\usephysicsmodule[showleft=3,showtop=3]{xmat}
\usepackage{derivative}
%% This is part of physics2 documentation.
%% ******************************************************
%% * This work may be distributed and/or modified under *
%% * the conditions of the LaTeX Project Public License *
%% *     http://www.latex-project.org/lppl.txt          *
%% * either version 1.3c of this license or any later   *
%% * version.                                           *
%% ******************************************************
\makeatletter
\usepackage[nohyperref]{doc}
\renewcommand*\l@subsection{\@dottedtocline{2}{1.5em}{2.1em}}
\def\contentsonly{\@starttoc{toc}}
\usepackage{manfnt}
\protected\def\pardanger{\ifnum\lastnodetype=1%
  \leavevmode\hbox to \z@{\hss\dbend\hskip10pt\hskip\parindent}\else
  \ifnum\lastnodetype=11%
  \leavevmode\hbox to \z@{\hss\dbend\hskip10pt}%
  \fi\fi\ignorespaces}
\usepackage{hyperref,color,xr}
\definecolor{pkgcolor}{RGB}{0,120,80}
\definecolor{moducolor}{RGB}{0,0,150}
\hypersetup{colorlinks,urlcolor=magenta,linkcolor=black,filecolor=blue,
  pdfstartview=FitH,pdfview=FitH,pdfcreator=XeTeX output}
\def\s@pkg#1{\texorpdfstring{\textcolor{pkgcolor}{\textsf{#1}}}{“#1”}}
\def\@pkg#1{\texorpdfstring{\href{https://www.ctan.org/pkg/#1}%
  {\textcolor{pkgcolor}{\textsf{#1}}}}{“#1”}}
\DeclareRobustCommand\pkg{\@ifstar\@pkg\s@pkg}
\def\modu#1{\texorpdfstring{\textcolor{moducolor}{\textsf{#1}}}{“#1”}}
\def\opt#1{\texorpdfstring{\texttt{#1}}{“#1”}}
\protected\def\marg#1{\texttt\{\meta{#1}\texttt\}}
\protected\def\oarg#1{\texttt[\meta{#1}\texttt]}
\parindent2.4em
\linespread{1.06}
\usepackage{multicol,fancyvrb,verbatim,array,hologo}
\fvset{xleftmargin=\parindent}
\newenvironment{displayed}{\trivlist\item[]\leftskip\parindent\relax}{\endtrivlist}
\usepackage[normalem]{ulem}
\newcounter{example}[subsection]
\renewcommand\theexample{\thesubsection.\arabic{example}}
\newwrite\example@out
\def\example@name{\jobname.example.aux}
\long\def\example@start{\begingroup\@bsphack
  \immediate\openout\example@out=\example@name
  \let\do\@makeother\dospecials\catcode`\^^M\active
  \def\verbatim@processline{\immediate\write\example@out{\the\verbatim@line}}%
  \verbatim@start}
\long\def\example@end{\immediate\closeout\example@out\@esphack\endgroup%
  \trivlist\item[]\relax
    \leavevmode\hbox to \z@{%
      \hbox to \z@{\hss{\footnotesize[\theexample]}\hskip10pt}%
      \hskip\parindent
      \begin{minipage}[c]{0.55\textwidth}%
        \small\verbatiminput{\example@name}%
      \end{minipage}%
      \fbox{%
        \begin{minipage}[c]{0.5\textwidth}%
          \normalsize\input{\example@name}%
        \end{minipage}%
      }%
    \hss}%
  \endtrivlist}
\newenvironment{example*}[1]%
  {\refstepcounter{example}\label{#1}\example@start}{\example@end}
\newenvironment{example}
  {\stepcounter{example}\example@start}{\example@end}
\author{Zhang Tingxuan}
\date{2023/04/02\quad Version 0.2.1\thanks{%
  \url{https://www.github.com/AlphaZTX/physics2}}}
\makeatother
\endinput

\title{\pkg{physics2} manual for the legacy \pkg{physics} users}
\begin{document}
\maketitle

\begin{abstract}
This short document describes \pkg*{physics2} package for those who are
used to the \pkg*{physics} package. This document is only a simple reference
manual for:
\begin{itemize}
\item Frequent users of the legacy \pkg{physics} package;
\item Those who have to maintain a document written with \pkg{physics};
\item Users who failed to use \pkg*{unicode-math} with \pkg{physics}.
\end{itemize}
It seems no reason for any other user to read \emph{this} document instead of
the \hyperref{./physics2.pdf}{}{}{package documentation} of \pkg{physics2},
because this document cannot describe the package in detail.

In this document, the modules of \pkg{physics2} will be introduced in
the same order as the \pkg{physics} documentation.
\end{abstract}

\section*{Contents}
\begin{multicols}{2}
\contentsonly
\end{multicols}
\ifnum\value{page}=1 \vfil\clearpage\fi

\section{Before you start}
\subsection{Legacy problems with \pkg{physics} package}
\label{subsec:physics-legacy-problems}
The \pkg{physics} package provides \cs{qty} command for automatic-sizing
braces. The \cs{qty} command would cause conflict with the \pkg{siunitx}
package, which provides a unified method to typeset numbers and units
correctly.

Besides, after you loaded \pkg{physics}, when you type \cs{homework} you
will get Maxwell equations and Schrödinger equation. The \cs{homework}
command is ``declared'' in \texttt{physics.sty} but it was not described
in the documentation. That is, if you have defined \cs{homework} before
loading \pkg{physics} package, \pkg{physics} would overwrite the definition
``silently''.

The vector-notation part of \pkg{physics} uses \pkg*{amsmath}'s (more exactly, 
\texttt{amsbsy.sty}'s) \cs{boldsymbol} command to generate bold vectors.
Commands for cross/dot product are defined with \cs{boldsymbol}. 
\cs{boldsymbol} uses \cs{mathversion}, a \LaTeXe\ kernel command that works
well with traditional TFM-based fonts but fails when using \pkg{unicode-math}.

In the definition of \cs{imat}, \cs{xmat}, \cs{dmat} and \cs{admat} commands
from \pkg{physics}, there is a \cs{newtoks} command which allocates a token
list register and two \cs{newcount} commands allocating two count registers.
Every time you write a command like \cs{imat} in your document, then one token
list register and two count registers will be wasted. What's even worse is
that, if you wrote really too many matrix commands from \pkg{physics} (for
example, 32767 \cs{imat}s in \hologo{LuaLaTeX}), there'd be no room for a new
\cs{count}.

\pkg{physics} integrated all the functions in one file (\texttt{physics.sty}),
that is, you cannot load one of the total seven parts of functions; you have
to load the seven parts altogether, even included the extra \cs{homework}
command we mentioned in the first paragraph.

Moreover, the code of \texttt{physics.sty} ``abuses'' the \texttt{g}-type
arguments of \pkg*{xparse} package. Therefore the syntax of \pkg{physics} package
looks kind of weird. See \hyperref{https://tex.stackexchange.com/questions/%
470819/macros-dv-and-pdv-eat-the-subsequent-parenthesis-argument/470842#470842}%
{}{}{here} for more.

\subsection{Loading \pkg{physics2}}
The \pkg{physics2} package includes different modules, among which every module
focuses on one single function.

Write the following line in the preamble to load \pkg{physics2}:
\begin{Verbatim}
\usepackage{physics2}
\end{Verbatim}
But this is not enough. \pkg{physics2} contains different modules, among which,
only the \modu{common} module would be loaded automatically by the package.
If you want to load other modules of \pkg{physics2}, write this after loading
\pkg{physics2} package:
\begin{displayed}
\cs{usephysicsmodule}\marg{module list}
\end{displayed}
For example, ``\verb|\usephysicsmodule{ab,doubleprod}|'' loads the \modu{ab}
module and the \modu{doubleprod} module.

You can also load a module with options:
\begin{displayed}
\cs{usephysicsmodule}\oarg{option list}\marg{module}
\end{displayed}
For example, ``\verb|\usephysicsmodule[legacy]{ab}|'' loads \modu{ab} with the
option ``\opt{legacy}''.

\pardanger Attention, if you used any font package in your document, remember
that \pkg{physics2} requires to be loaded \emph{after} font packages.


\section{List of commands}
\subsection{Automatic bracing}\label{subsec:ab-and-legacy}
As mentioned in \S\ref{subsec:physics-legacy-problems}, the \cs{qty} command
from \pkg{physics} would cause conflicts with \pkg{siunitx}. The command for
automatic braces in \pkg{physics2} is \cs{ab}, a shorthand for 
{\bfseries a}utomatic {\bfseries b}races.

The \cs{ab} command requires the \modu{ab} module, so don't forget to write
\verb|\usephysicsmodule{ab}| in the preamble after you loaded \pkg{physics2}.
Always remember, \emph{do not put an \cs{ab} separately in the end of a math
formula}. Take some examples:
\begin{example}
\[ \ab ( \frac12 )  \quad
   \ab [ \frac12 ]  \quad
   \ab\{ \frac12 \} \]
\end{example}
\cs{ab} can modify a delimiter-braced subformula. But the delimiters should
not be out of the range described by the following chart:
\begin{center}
\begin{tabular}{c@{\hskip2em}l@{\hskip2em}c}
\opt{(},\quad\opt{)} && \\
\opt{[},\quad\opt{]} && \\
\cs{\{},\quad\cs{\}} &or& \cs{lbrace},\quad\cs{rbrace} \\
\opt{<},\quad\opt{>} &or& \cs{langle},\quad\cs{rangle} \\
\opt{|},\quad\opt{|} &or&   \cs{vert},\quad\cs{vert} \\
 \cs{|},\quad\cs{|}  &or&   \cs{Vert},\quad\cs{Vert}
\end{tabular}
\end{center}
For example, \verb|$\ab{foo}$| and \verb|$\ab(foo]$| are illegal, but
\verb|$\ab\{foo\}$| and \verb|$\ab(foo)$| are okay; \verb|$\ab([)$|
is okay but \verb|$\ab(()$| is illegal.

\pardanger Attention, if you want to delimit a subformula with ``\{'' and
``\}'', you can only write \verb|\{|, \verb|\}| or \cs{lbrace}, \cs{rbrace}
around it. \verb|{| and \verb|}| are not supported in \modu{ab} module.

Between \cs{ab} and the first delimiter can be a ``biggg'' command, that is,
from \cs{big} to \cs{Bigg}. Actually, you can also write \cs{biggg} and
\cs{Biggg} because \pkg{physics2} defines these after you load it. For example,
\begin{example}
\[ \ab\Big \| \frac12 \| \quad
   \ab\Bigg < \frac12 >  \quad
   \ab\Biggg| \frac12 |  \]
\end{example}
Between \cs{ab} and the first delimiter can also be a star (\opt{*}), which
means ``use the default size of delimiters''. But in this situation, you
needn't use the \cs{ab} command at all.

The \pkg{physics} package provides commands like \cs{pqty}, \cs{bqty}. In
the \modu{ab} module of \pkg{physics2}, these commands have changed to
\cs{pab}, \cs{bab}, etc. The following example shows all the 
\texttt{\textbackslash}$X$\texttt{ab} commands in \modu{ab} module:
\begin{example}
\def\0{\frac12}
\[ \pab{\0} \quad \bab{\0}
            \quad \Bab{\0} \]
\[ \aab{\0} \quad \vab{\0}
            \quad \Vab{\0} \]
\end{example}
\texttt{\textbackslash$X$ab} can take an optional star and an optional \oarg{biggg}
argument. For example,
\begin{example}
\def\0{\frac12}
\[ \pab[Big]{\0} \quad \bab*{\0} \]
\end{example}

\pkg{physics} also provides the following commands:
\begin{Verbatim}[fontsize=\small]
\abs   \norm   \eval   \order      \comm   \acomm   \pb
\end{Verbatim}
\pardanger These commands are not originally supported by \pkg{physics2}, but
the first four commands can be used through the \modu{ab.legacy} module of
\pkg{physics2}:
\begin{Verbatim}
\usephysicsmodule{ab.legacy}
\end{Verbatim}
For example,
\begin{example}
\def\0{\frac12}
\[ \abs{\0} \quad \abs[big]{\0}
            \quad \abs*{\0}    \]
\end{example}
Users of the legacy \pkg{physics} package should notice that the syntax of
\cs{eval} has been changed. The \modu{ab.legacy} module abandoned the 
\verb"\eval(foo|"-like syntax. The new \cs{eval}'s syntax is just like other
commands in this module. There are also two variants of \cs{eval} --- \cs{peval}
and \cs{beval}. For example,
\begin{example}
\def\0{1+\frac12x}
\[ \eval{\0}_a^b      \quad
   \peval*{\0}_a^b    \quad
   \beval[big]{\0}_a^b   \]
\end{example}
The \cs{comm}, \cs{acomm} and \cs{pb} (Poisson bracket) are not supported.
But you can write like \verb|\ab[foo,baz]| or \verb|\bab{foo,baz}| instead.

By the way, you can set the ``order'' symbol in \modu{ab.legacy} through
the \opt{order} option like this:
\begin{Verbatim}
\usephysicsmodule[order=O]{ab.legacy}
\end{Verbatim}
Then \verb|$\order(N)$| yields $O(N)$.

\subsection{Vector notation}
Unfortunately, there is not a plan for \pkg{physics2} to support this part
of \pkg{physics} completely, but the rest of this section will show some
methods to maintain the document written with \pkg{physics}.

The \cs{vb}(\opt{*}), \cs{va}(\opt{*}) and \cs{vu}(\opt{*}) are not supported
in any module of \pkg{physics2}. But these commands can be defined by copying
the following lines below and pasting them in the preamble:
\begin{Verbatim}[fontsize=\small]
\makeatletter
\newcommand\vb{\@ifstar\boldsymbol\mathbf}
\newcommand\va[1]{\@ifstar{\vec{#1}}{\vec{\mathrm{#1}}}}
\newcommand\vu[1]{%
  \@ifstar{\hat{\boldsymbol{#1}}}{\hat{\mathbf{#1}}}}
\makeatother
\end{Verbatim}
The \cs{boldsymbol} command requires the \pkg*{amsmath} or \pkg*{bm} package.
If you prefer to use \pkg{bm}, you can also use the \cs{bm} command.
What's more, if you tried the commands above, you might find that,
the result of \cs{va} above is different from that of \pkg{physics}.
This is because, if you choose to present a vector in bold, there's almost
no need to put a \cs{vec} ($\vec{\mskip9mu}$) sign above it.

However, the method above may not work well with \pkg*{unicode-math}
because there are so many OpenType math fonts without a bold version.
When using \pkg{unicode-math}, it's recommended to use \cs{symbf} and
\cs{symbfit} for a separate vector. For example, \verb|$\symbf{0}$| yields
$\symbf{0}$, and \verb|$\symbfit{A}$| yields $\symbfit{A}$.

The \cs{vdot} and \cs{cross} commands are not supported in any module
of \pkg{physics2}. Actually, there is no need to use a bold ``$\cdot$''
or ``$\times$'' for the products of two vectors. Using \cs{cdot} and
\cs{times} is enough.

The commands related to ``$\nabla$'' are supported through \modu{nabla.legacy}
module. These commands are \cs{grad}, \cs{div} and \cs{curl}. These commands
should not be put in the end of a math formula either (just like \cs{ab}).
Notice that the former \cs{div} command for a ``$\divsymbol$'' (if there
exists one) is redefined as \cs{divsymbol}. For example,
\begin{example}
% \usephysicsmodule{nabla.legacy}
\[ \grad F      \quad
   \grad(\frac G2)     \]
\[ \div\Bigg[X]  \quad
    \curl*\{\frac Y2\} \]
\[ 2 \divsymbol 1      \]
\end{example}
Actually, the nabla-related commands end with \cs{ab}. Thus, the subformula
after these commands can be delimited with \verb|()|, \verb|[]| and \verb|\{\}|.

The \modu{nabla.legacy} requires the \pkg*{fixdif} package at least
version 2.0 (file date: 2023/01/31 or after 2023/01/31).

By the way, if you are used to writing \cs{bm} for a vector but interested in
\pkg{unicode-math}, the \modu{bm-um.legacy} module would be a passable
alternative to \pkg{bm} package. Notice that the \cs{bm} command from the
\modu{bm-um.legacy} module can only take \emph{one} letter (or \emph{one} digit)
as its argument.

\subsection{Operators}
There's no plan for \pkg{physics2} to support this part of \pkg{physics}
completely. The syntax in this part of \pkg{physics} (such as \verb|\tan[2](x)|)
abuses \pkg*{xparse}.

It's suggested to write like this if you used the \modu{ab} module:
\begin{Verbatim}[fontsize=\small]
$ \sin^2 \ab( \frac{\alpha}{2} ) $
\end{Verbatim}
Rather than take the superscript as an optional argument of the command of
log-like functions.

The \pkg{physics} package provides a bundle of commands for log-like functions
that have not been defined in the \LaTeXe\ kernel. Those log-like functions
can be used with the \modu{op.legacy} module; this module does not support
the syntax of \pkg{physics} either. For example:
\begin{example}
% \usephysicsmodule{op.legacy}
\[ \asin^2 x \quad \rank \{ A \} \]
\end{example}
The \cs{Re} and \cs{Im} commands are redefined as operators ``$\Re$'' and
``$\Im$'', while $\Resymbol$ and $\Imsymbol$ are reserved as \cs{Resymbol} and
\cs{Imsymbol}. $\Resymbol$ and $\Imsymbol$ are ordinary symbols but $\Re$ and
$\Im$ are operators.

\subsection{Quick quad text}\label{subsec:qtext}
The \modu{qtext.legacy} module provides the \cs{q}\meta{foo} commands
for \cs{quad}-wrapped texts. These commands have the same syntax as
\pkg{physics}. For example,
\begin{example}
% \usephysicsmodule{qtext.legacy}
\[ A \qq {foo bar} B \]
\[ A \qq*{foo bar} B \]
\[ C \qcc D \qcc*  E \]
\[ F \qif G \qthen H \]
\end{example}
All the commands described in \S2.4 of
\hyperref{http://mirrors.ctan.org/macros/latex/contrib/physics/physics.pdf}{}{}%
{\textsf{physics} documentation} are supported when using \modu{qtext.legacy}
module, but I don't recommend using this module unless you are maintaining a
document written with \pkg{physics}'s \cs{q}\meta{foo} commands.

\subsection{Derivatives}
There is no plan for \pkg{physics2} to support this part of \pkg{physics}. If
you want to typeset the differential operators on a better sense, you can try
the \pkg*{fixdif} package; if you want an easy way to type derivatives, you
can try the \pkg*{derivative} package. These two packages can be used
together. For example,
\begin{example}
% \usepackage{fixdif,derivative}
\[ \pdv{f}{x,y,z} \d x \]
Math ($\d x$) v.s.\ Text (\d x)
\end{example}
Here are the documentations of
\hyperref{http://mirrors.ctan.org/macros/latex/contrib/fixdif/fixdif.pdf}%
{}{}{\textsf{fixdif}} and
\hyperref{http://mirrors.ctan.org/macros/latex/contrib/derivative/derivative%
.pdf}{}{}{\textsf{derivative}}.

\pkg{fixdif}'s commands behave better in superscripts and subscripts.

\subsection{Dirac bra-ket notation}
There are two solutions to Dirac bra-ket in \pkg{physics2} --- \modu{ab.braket}
and \modu{braket}. These two modules are \emph{not} compatible and neither of
them supports \pkg{physics}'s syntax completely.
Click \hyperlink{para:ab.braket}{\textcolor{cyan}{here}} to see the
\modu{ab.braket} module and \hyperlink{para:braket}{\textcolor{cyan}{here}}
to see the \modu{braket} module.

\paragraph{The \modu{ab.braket} module}
\hypertarget{para:ab.braket}{This} module provides four commands --- \cs{bra}, \cs{ket}, \cs{braket} and
\cs{ketbra}. After these commands can be a star (\opt{*}) or a ``biggg''
command. These commands share similar syntaxes like \cs{ab}'s syntax. But,
\emph{the bra-ket commands from \modu{ab.braket} module are completely different
from \cs{ab}}. Their internal structures are different.

The argument of \cs{bra} should be delimited with \opt{<} and \opt{|}, that is,
\begin{center}
\cs{bra} \opt{<} \meta{subformula} \opt{|}
\end{center}
For example,
\begin{example}
\[ \bra < \frac \phi 2 | \]
\[ \bra*< \frac \phi 2 | \]
\[ \bra\Big< \phi  |     \]
\end{example}
The argument of \cs{ket} should be delimited with \opt{|} and \opt{>}, that is,
\begin{center}
\cs{ket} \opt{|} \meta{subformula} \opt{>}
\end{center}
For example,
\begin{example}
\[ \ket | \frac \psi 2 > \]
\[ \ket*| \frac \psi 2 > \]
\[ \ket\Big| \psi >      \]
\end{example}
\pardanger
If you want to write ``$>$'' and ``$<$'' for relations in the argument of
\cs{bra} and \cs{ket}, you can write \verb|\mathrel{>}| and \verb|\mathrel{<}|
(although there is almost no such need).

The argument of \cs{braket} should be delimited with \opt{<} and \opt{>},
that is,
\begin{center}
\cs{braket} \opt{<} \meta{subformula} \opt{>}
\end{center}
In the \meta{subformula} argument, every ``\opt{|}'' will be regarded as an
extensible vertical bar. For example,
\begin{example}
\[ \braket< \phi >            \]
\[ \braket< \phi | \psi >     \]
\[ \braket< \phi | A | \psi > \]
\end{example}
\begin{example}
\def\0{\frac\phi2}
\[ \braket     < \0 | \psi >  \]
\[ \braket*    < \0 | \psi >  \]
\[ \braket\Bigg< \0 | \psi >  \]
\end{example}
The argument of \cs{ketbra} should be delimited with \opt{|} and \opt{|}.
In the argument, \opt{>} and \opt{<} will be regarded as extensible $\rangle$
and $\langle$. That is,
\begin{center}
\cs{ketbra} \opt{|} \meta{subformula$_1$} \opt{>} \meta{optional}
  \opt{<} \meta{subformula$_2$} \opt{|}
\end{center}
For example,
\begin{example}
\def\0{\frac\phi2}
\[ \ketbra     | \0 >< \psi | \]
\[ \ketbra*    | \0 >< \psi | \]
\[ \ketbra\Bigg| \0 >< \psi | \]
\end{example}
\begin{example}
\def\0{\frac\phi2}
\[ \ketbra| \0 >_x^y < \psi | \]
\end{example}
\pardanger
If you want to write ``$>$'' and ``$<$'' for relations in the argument of
\cs{braket} and \cs{ketbra}, you can write \cs{>} and \cs{<} (although there
is almost no such need). It is quite different from \verb|\mathrel{>}| or
\verb|\mathrel{<}| because in these commands' argument, \opt{>} and \opt{<}
will be redefined.

\paragraph{The \modu{braket} module}
\begingroup
\makeatletter\def\phy@requiremodule#1{}%
\def\PackageWarning#1#2{}%
\input phy-braket.sty\makeatother
\hypertarget{para:braket}{This} module contains four commands --- \cs{bra},
\cs{ket}, \cs{braket} and \cs{ketbra}. After these commands can be a star
(\opt{*}) or a square-bracket-delimited size option, the size option can
take the following values:
\begin{center}
\opt{big},\quad\opt{Big},\quad\opt{bigg},\quad\opt{Bigg},\quad
\opt{biggg}\quad or\quad\opt{Biggg}.
\end{center}
Star stands for ``do not size the bra-ket automatically''.

The argument(s) of these four commands are braced with \verb|{| and \verb|}|.
\cs{bra} and \cs{ket} take one mandatory argument. For example,
\begin{example}
\def\0{\frac\phi2}
\[ \bra {\0} \quad \bra* {\0}
             \quad \bra[Big] {\0} \]
\[ \ket {\0} \quad \ket* {\0}
             \quad \ket[Big] {\0} \]
\end{example}
The \cs{braket} command, in default, can take two arguments.
\begin{example}
\def\0{\frac\phi2}
\[ \braket {\0} {\psi}   \quad
   \braket*{\0} {\psi}   \quad
   \braket[big] {\0} {\psi} \]
\end{example}
If you want \cs{braket} to take one or three arguments, you can write the
number of arguments in the square bracket. If you need to specify the size
of bra-ket simultaneously, you need to separate the number and the size with
a comma:
\begin{example}
\def\0{\frac\phi2}
\[ \braket [1] {\0} \quad
   \braket*[1] {\0} \]
\[ \braket [3] {\0}{A}{\psi}    \]
\[ \braket[3,big] {\0}{A}{\psi}
     \quad
   \braket[Big,3] {\0}{A}{\psi} \]
\end{example}
The \cs{ketbra} command takes two mandatory arguments. It can also take an
optional argument between the two mandatory arguments. The optional argument
will be placed between $\rangle$ and $\langle$:
\begin{example}
\def\0{\frac\phi2}
\[ \ketbra  {\0} {\psi}    \quad
   \ketbra* {\0} {\psi}       \]
\[ \ketbra [Bigg] {\0} {\psi} \]
\[ \ketbra {\0} [_x^y] {\psi} \]
\end{example}
\endgroup

\subsection{Matrix macros}
Unfortunately, \pkg{physics2} do not support the \cs{mqty} command from
\pkg{physics}. If you are used to this command, you can write like this:
\begin{Verbatim}[fontsize=\small]
\newcommand\mqty[1]{\begin{matrix}#1\end{matrix}}
\newcommand\pmqty[1]{\begin{pmatrix}#1\end{pmatrix}}
$\ab(\mqty{foo})$  or  $\pmqty{foo}$
\end{Verbatim}
These are equal to physics's \verb|\mqty(foo)| (require \pkg{amsmath}).

\pkg{physics2}'s \modu{diagmat} module provides \cs{diagmat} command
for diagonal matrices. For example,
\begin{example}
\[
  \diagmat { 1, 2, 3 }
\]
\end{example}
\begin{example}
\[
  \pdiagmat [ empty = {} ]
    { a, b, c, d }
\]
\end{example}
\cs{pdiagmat}, \cs{bdiagmat}, \cs{Bdiagmat}, \cs{vdiagmat} and \cs{Vdiagmat}
are also available.

\pkg{physics2}'s \modu{xmat} module provides \cs{xmat} command
for matrices with formatted entries. For example,
\begin{example}
\[
  \xmat{a}{2}{3}
\]
\end{example}
\begin{example}
% \usephysicsmodule
%   [showleft=3,showtop=3]{xmat}
\[
  \pxmat{X}{m}{n}
\]
\end{example}
\begin{example}
\[
  \xmat [showleft=2,showtop=2,
    format=\texttt{#1[#2][#3]}]
    {x}{m}{n}
\]
\end{example}
\cs{pxmat}, \cs{bxmat}, \cs{Bxmat}, \cs{vxmat} and \cs{Vxmat} are also available.









\end{document}